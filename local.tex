% \usepackage{cite}
% \usepackage[numbers,sort]{natbib}
\usepackage{amsmath,amsfonts}
\usepackage{algorithmic}
\usepackage{graphicx}
\usepackage{textcomp}
% \usepackage[hyphens]{url}
\usepackage[linesnumbered,ruled,vlined]{algorithm2e}
\usepackage[utf8]{inputenc}
\newcommand\commentformat[1]{\small\ttfamily{#1}}
\SetCommentSty{commentformat}
\SetKwComment{Comment}{// }{}
\usepackage{array}
\usepackage{tabularx}
\usepackage{stfloats}
\usepackage{url}
\usepackage{makecell}
\usepackage{multicol}
\usepackage{multirow}
\usepackage{float}
\usepackage{placeins}
\usepackage{mathrsfs}
\usepackage{dutchcal}
\usepackage{amsthm}
\usepackage{threeparttable}

\let\oldbaselinestretch\baselinestretch
\usepackage[UTF8]{ctex}
\let\baselinestretch\oldbaselinestretch

% 定义新定理样式:缩进1em,标题斜体,内容正常
\newtheoremstyle{indenteditalic}
  {3pt}   % 上方间距
  {3pt}   % 下方间距
  {}      % 主体字体(正常)
  {1em}   % 缩进量
  {\itshape} % 标题字体(斜体)
  {.}     % 标题后标点
  {0.5em} % 标题与内容间距
  {}      % 定理头格式(可选)

% 应用新样式到definition环境
\theoremstyle{indenteditalic}
\newtheorem{definition}{Definition}[section]
% 强制声明字体尺寸兼容性(添加到导言区)
% \DeclareFontFamily{U}{rsfs}{\skewchar\font127}
% \DeclareFontShape{U}{rsfs}{m}{n}{<-> s*[1.0] rsfs10}{} % 缩放系数可调

\usepackage{subcaption}
\usepackage{color}
\usepackage{import}
\usepackage{listings}
\usepackage{booktabs}
\usepackage{url}
\usepackage{hyperref}
\usepackage{fancyhdr} 
\usepackage{adjustbox}
% \usepackage{amssymb} 
% \newtheorem{definition}{Definition}
% \newtheorem{assumption}{Assumption}
% \newcommand{\definitionautorefname}{Definition}
% \newcommand{\assumptionautorefname}{Assumption}
\PassOptionsToPackage{dvipsnames}{xcolor}
\usepackage[dvipsnames]{xcolor}
% \usepackage{pdfpages}
\colorlet{RED}{red}
% \usepackage{subfig}
% \setstretch{0.95} % 或者更小的值
% \usepackage{setspace}
% \setlength{\parskip}{0pt}
% \usepackage{titlesec}
% \titlespacing{\section}{0pt}{2pt}{2pt}  % 控制章节标题的上、下间距
% \titlespacing{\subsection}{0pt}{1pt}{1pt}  % 控制小节标题的上、下间距

% \setlength{\textfloatsep}{5pt}   % 控制图表与正文之间的垂直距离
% \setlength{\floatsep}{5pt}       % 控制两个图表之间的距离
% \setlength{\intextsep}{5pt}      % 控制浮动图表与正文之间的距离
% \setlength{\abovecaptionskip}{2pt}  % 调整标题与图表之间的距离
% \setlength{\belowcaptionskip}{1pt}  % 调整标题与正文之间的距离
\usepackage{cleveref}
\usepackage{tikz}
\usepackage{alltt}

\usepackage{algorithm2e}
% \usepackage{algorithm}
% \usepackage{algpseudocode}
% \usepackage{CJKutf8} % Commented out: conflicts with ctex's xeCJK
\usepackage{arydshln} % Add this package for dashed lines

% 颜色定义
\definecolor{codebg}{RGB}{255,255,255}
\definecolor{keywordcolor}{RGB}{163,21,21}
\definecolor{commentcolor}{RGB}{0,100,80}
\definecolor{stringcolor}{RGB}{163,21,21}
\definecolor{codefont}{RGB}{0,0,0} % 亮黑色字体
\definecolor{typecolor}{RGB}{0,150,0}

\newcommand{\hlcode}[1]{\colorbox{red!80}{\textcolor{white}{#1}}}
% \newcommand{\corecode}[1]{%
%   \colorbox{blue!20}{\textbf{#1}}%
% }
\newcommand{\corecodecomment}[1]{%
  \colorbox{blue!70}{\textcolor{white}{#1}}%
}

% 设置 listings 样式
\lstset{
  backgroundcolor=\color{codebg},
  basicstyle=\small\ttfamily\color{codefont},
  % fontfamily=Fira Code,
  keywordstyle=\color{keywordcolor},
  commentstyle=\color{commentcolor}\itshape,
  stringstyle=\color{stringcolor},
  numberstyle=\tiny\color{gray},
  numbers=left,
  numbersep=5pt,
  tabsize=2,
  showstringspaces=false,
  breaklines=true,
  prebreak=\mbox{\textcolor{gray}{\tiny$\hookleftarrow$}\space}, % 将换行符显示在上一行末尾
  frame=none,
  escapeinside={+@}{@+},
  captionpos=b,
  language=C++,
  morekeywords={align,define,declare,call,ret,br,phi,load,store,add,sub,mul,icmp,fcmp,bitcast,ptrtoint,inttoptr,alloca,getelementptr,select,unreachable},
  alsoletter={\%,@},
  morekeywords={[2]\%[a-zA-Z_][a-zA-Z0-9_]*,@[a-zA-Z_][a-zA-Z0-9_]*},
  keywordstyle={[2]\color{blue!70}},
  emph={TilingData,Status,T,BaseParams,SingleCoreParams,TData,OpTiling,TilingAlgo,AlgoBase,Algo1,Algo2,PartData,assert}, % 添加自定义类型名
  emphstyle=\color{typecolor}, % 设置类型名高亮样式
  moredelim=**[is][\colorbox{blue!20}]{@HL@}{@HL@},
  mathescape=true,
  aboveskip=0.5em, % 控制代码块上方的距离
  belowskip=0em  % 控制代码块下方的距离
}

\newcommand{\myparagraph}[1]{\noindent\textbf{#1}}
% \usepackage[numbers]{natbib}
\usepackage[mathcal]{euscript}
\renewcommand\lstlistingname{Code}
\usepackage{ulem}
\usepackage{pifont}
\usepackage{array}
\usepackage[table]{xcolor} % 引入xcolor包并启用表格颜色支持
\definecolor{lightgray}{rgb}{0.9,0.9,0.9} % 自定义浅灰色
\long\def\com#1{}
\long\def\xxx#1{{\bf XXX: }{\small [#1]}}
\long\def\abbr#1#2{#2}			% long version
\def\bibbrev#1#2{#2}			% long version\newcommand{\abcite}[2]{\abbr{\cite{#1}}{\cite{#1,#2}}}
\newcommand{\bibconf}[3][]{#1 #2}
\newcommand{\mysect}[1]{\textit{\textbf{#1}}} 
\newcommand{\kw}[1]{{\it #1}} % keyword
\newcommand{\kn}[1]{\texttt{\small #1}}	% keyword
\newcommand{\kd}[1]{\textbf{#1}} % keyword
\newcommand{\KW}[1]{{\it #1}} % keyword
\newcommand{\KN}[1]{\texttt{\small #1}}	% keyword
\newcommand{\KD}[1]{\textbf{#1}} % keyword
\newcommand{\ksf}[1]{{\small \textsf{#1}}} % small textsf
\newcommand{\codet}[1]{{\small \textsf{#1}}} % small textsf
\newcommand\etal{{\it{et al.\ }}}
\newcommand\eg{{\it{e.g.,\ }}}
\newcommand\ie{{\it{i.e.,\ }}}
\newcommand\etc{{\it{etc.\ }}}

\newcommand{\annotate}[1]{\textcolor{blue}{#1} \\}
\newcommand{\todo}[1]{\textcolor{}{#1}}
\newcommand{\leftblank}{\textcolor{}{XXX}\xspace}
\newcommand{\mysys}{TilingInfer\xspace}
\newcommand{\hlight}[1]{\textcolor{red}{#1}}
