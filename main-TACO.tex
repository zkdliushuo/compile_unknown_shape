%%
%% This is file `sample-manuscript.tex',
%% generated with the docstrip utility.
%%
%% The original source files were:
%%
%% samples.dtx  (with options: `all,proceedings,bibtex,manuscript')
%% 
%% IMPORTANT NOTICE:
%% 
%% For the copyright see the source file.
%% 
%% Any modified versions of this file must be renamed
%% with new filenames distinct from sample-manuscript.tex.
%% 
%% For distribution of the original source see the terms
%% for copying and modification in the file samples.dtx.
%% 
%% This generated file may be distributed as long as the
%% original source files, as listed above, are part of the
%% same distribution. (The sources need not necessarily be
%% in the same archive or directory.)
%%
%%
%% Commands for TeXCount
%TC:macro \cite [option:text,text]
%TC:macro \citep [option:text,text]
%TC:macro \citet [option:text,text]
%TC:envir table 0 1
%TC:envir table* 0 1
%TC:envir tabular [ignore] word
%TC:envir displaymath 0 word
%TC:envir math 0 word
%TC:envir comment 0 0
%%
%% The first command in your LaTeX source must be the \documentclass
%% command.
%%
%% For submission and review of your manuscript please change the
%% command to \documentclass[manuscript, screen, review]{acmart}.
%%
%% When submitting camera ready or to TAPS, please change the command
%% to \documentclass[sigconf]{acmart} or whichever template is required
%% for your publication.
%%
%%
% \documentclass[manuscript,screen,review,numbers,sort&compress,hyphens]{acmart}
\documentclass[manuscript,screen,review]{acmart}
% \documentclass[sigconf,screen,review]{acmart}
\citestyle{acmnumeric}

% \usepackage{cite}
% \usepackage[numbers,sort]{natbib}
\usepackage{amsmath,amsfonts}
\usepackage{algorithmic}
\usepackage{graphicx}
\usepackage{textcomp}
% \usepackage[hyphens]{url}
\usepackage[linesnumbered,ruled,vlined]{algorithm2e}
\usepackage[utf8]{inputenc}
\newcommand\commentformat[1]{\small\ttfamily{#1}}
\SetCommentSty{commentformat}
\SetKwComment{Comment}{// }{}
\usepackage{array}
\usepackage{tabularx}
\usepackage{stfloats}
\usepackage{url}
\usepackage{makecell}
\usepackage{multicol}
\usepackage{multirow}
\usepackage{float}
\usepackage{placeins}
\usepackage{mathrsfs}
\usepackage{dutchcal}
\usepackage{amsthm}
\usepackage{threeparttable}

\let\oldbaselinestretch\baselinestretch
\usepackage[UTF8]{ctex}
\let\baselinestretch\oldbaselinestretch

% 定义新定理样式:缩进1em,标题斜体,内容正常
\newtheoremstyle{indenteditalic}
  {3pt}   % 上方间距
  {3pt}   % 下方间距
  {}      % 主体字体(正常)
  {1em}   % 缩进量
  {\itshape} % 标题字体(斜体)
  {.}     % 标题后标点
  {0.5em} % 标题与内容间距
  {}      % 定理头格式(可选)

% 应用新样式到definition环境
\theoremstyle{indenteditalic}
\newtheorem{definition}{Definition}[section]
% 强制声明字体尺寸兼容性(添加到导言区)
% \DeclareFontFamily{U}{rsfs}{\skewchar\font127}
% \DeclareFontShape{U}{rsfs}{m}{n}{<-> s*[1.0] rsfs10}{} % 缩放系数可调

\usepackage{subcaption}
\usepackage{color}
\usepackage{import}
\usepackage{listings}
\usepackage{booktabs}
\usepackage{url}
\usepackage{hyperref}
\usepackage{fancyhdr} 
\usepackage{adjustbox}
% \usepackage{amssymb} 
% \newtheorem{definition}{Definition}
% \newtheorem{assumption}{Assumption}
% \newcommand{\definitionautorefname}{Definition}
% \newcommand{\assumptionautorefname}{Assumption}
\PassOptionsToPackage{dvipsnames}{xcolor}
\usepackage[dvipsnames]{xcolor}
% \usepackage{pdfpages}
\colorlet{RED}{red}
% \usepackage{subfig}
% \setstretch{0.95} % 或者更小的值
% \usepackage{setspace}
% \setlength{\parskip}{0pt}
% \usepackage{titlesec}
% \titlespacing{\section}{0pt}{2pt}{2pt}  % 控制章节标题的上、下间距
% \titlespacing{\subsection}{0pt}{1pt}{1pt}  % 控制小节标题的上、下间距

% \setlength{\textfloatsep}{5pt}   % 控制图表与正文之间的垂直距离
% \setlength{\floatsep}{5pt}       % 控制两个图表之间的距离
% \setlength{\intextsep}{5pt}      % 控制浮动图表与正文之间的距离
% \setlength{\abovecaptionskip}{2pt}  % 调整标题与图表之间的距离
% \setlength{\belowcaptionskip}{1pt}  % 调整标题与正文之间的距离
\usepackage{cleveref}
\usepackage{tikz}
\usepackage{alltt}

\usepackage{algorithm2e}
% \usepackage{algorithm}
% \usepackage{algpseudocode}
% \usepackage{CJKutf8} % Commented out: conflicts with ctex's xeCJK
\usepackage{arydshln} % Add this package for dashed lines

% 颜色定义
\definecolor{codebg}{RGB}{255,255,255}
\definecolor{keywordcolor}{RGB}{163,21,21}
\definecolor{commentcolor}{RGB}{0,100,80}
\definecolor{stringcolor}{RGB}{163,21,21}
\definecolor{codefont}{RGB}{0,0,0} % 亮黑色字体
\definecolor{typecolor}{RGB}{0,150,0}

\newcommand{\hlcode}[1]{\colorbox{red!80}{\textcolor{white}{#1}}}
% \newcommand{\corecode}[1]{%
%   \colorbox{blue!20}{\textbf{#1}}%
% }
\newcommand{\corecodecomment}[1]{%
  \colorbox{blue!70}{\textcolor{white}{#1}}%
}

% 设置 listings 样式
\lstset{
  backgroundcolor=\color{codebg},
  basicstyle=\small\ttfamily\color{codefont},
  % fontfamily=Fira Code,
  keywordstyle=\color{keywordcolor},
  commentstyle=\color{commentcolor}\itshape,
  stringstyle=\color{stringcolor},
  numberstyle=\tiny\color{gray},
  numbers=left,
  numbersep=5pt,
  tabsize=2,
  showstringspaces=false,
  breaklines=true,
  prebreak=\mbox{\textcolor{gray}{\tiny$\hookleftarrow$}\space}, % 将换行符显示在上一行末尾
  frame=none,
  escapeinside={+@}{@+},
  captionpos=b,
  language=C++,
  morekeywords={align,define,declare,call,ret,br,phi,load,store,add,sub,mul,icmp,fcmp,bitcast,ptrtoint,inttoptr,alloca,getelementptr,select,unreachable},
  alsoletter={\%,@},
  morekeywords={[2]\%[a-zA-Z_][a-zA-Z0-9_]*,@[a-zA-Z_][a-zA-Z0-9_]*},
  keywordstyle={[2]\color{blue!70}},
  emph={TilingData,Status,T,BaseParams,SingleCoreParams,TData,OpTiling,TilingAlgo,AlgoBase,Algo1,Algo2,PartData,assert}, % 添加自定义类型名
  emphstyle=\color{typecolor}, % 设置类型名高亮样式
  moredelim=**[is][\colorbox{blue!20}]{@HL@}{@HL@},
  mathescape=true,
  aboveskip=0.5em, % 控制代码块上方的距离
  belowskip=0em  % 控制代码块下方的距离
}

\newcommand{\myparagraph}[1]{\noindent\textbf{#1}}
% \usepackage[numbers]{natbib}
\usepackage[mathcal]{euscript}
\renewcommand\lstlistingname{Code}
\usepackage{ulem}
\usepackage{pifont}
\usepackage{array}
\usepackage[table]{xcolor} % 引入xcolor包并启用表格颜色支持
\definecolor{lightgray}{rgb}{0.9,0.9,0.9} % 自定义浅灰色
\long\def\com#1{}
\long\def\xxx#1{{\bf XXX: }{\small [#1]}}
\long\def\abbr#1#2{#2}			% long version
\def\bibbrev#1#2{#2}			% long version\newcommand{\abcite}[2]{\abbr{\cite{#1}}{\cite{#1,#2}}}
\newcommand{\bibconf}[3][]{#1 #2}
\newcommand{\mysect}[1]{\textit{\textbf{#1}}} 
\newcommand{\kw}[1]{{\it #1}} % keyword
\newcommand{\kn}[1]{\texttt{\small #1}}	% keyword
\newcommand{\kd}[1]{\textbf{#1}} % keyword
\newcommand{\KW}[1]{{\it #1}} % keyword
\newcommand{\KN}[1]{\texttt{\small #1}}	% keyword
\newcommand{\KD}[1]{\textbf{#1}} % keyword
\newcommand{\ksf}[1]{{\small \textsf{#1}}} % small textsf
\newcommand{\codet}[1]{{\small \textsf{#1}}} % small textsf
\newcommand\etal{{\it{et al.\ }}}
\newcommand\eg{{\it{e.g.,\ }}}
\newcommand\ie{{\it{i.e.,\ }}}
\newcommand\etc{{\it{etc.\ }}}

\newcommand{\annotate}[1]{\textcolor{blue}{#1} \\}
\newcommand{\todo}[1]{\textcolor{}{#1}}
\newcommand{\leftblank}{\textcolor{}{XXX}\xspace}
\newcommand{\mysys}{TilingInfer\xspace}
\newcommand{\hlight}[1]{\textcolor{red}{#1}}

\newcommand{\yu}[1] {{\textcolor{OliveGreen}{\textbf{#1}}}}
\newcommand{\yunote}[1] {{$\langle${\textcolor{red}{YU:\textbf{#1}}}$\rangle$}}
\newcommand{\ls}[1] {{$\langle${\textcolor{brown}{LS:\textbf{#1}}}$\rangle$}}
\newcommand{\yi}[1] {{$\langle${\textcolor{blue}{HYT:\textbf{#1}}}$\rangle$}}
\newcommand{\yiR}[2]{{\remv{#1}\yi{#2}}}
\newcommand{\yiRemv}[1]{{\color{magenta}\sout{#1}}}
\newcommand\sxy[1]{{$\langle${\textcolor{red}{SXY:\textbf{#1}}}$\rangle$}}
\newcommand\yzj[1]{{$\langle${\textcolor{red}{YZJ:\textbf{#1}}}$\rangle$}}
\newcommand\scnote[1]{{$\langle${\textcolor{red}{SC:\textbf{#1}}}$\rangle$}}
\newcommand\red[1] {{\textcolor{red}{#1}}}
\newcommand{\lsR}[2]{{{\color{magenta}\sout{#1}}#2}}

\newcommand\sR{{\mathbb{R}}}
\newcommand\points{{\mathbb{E}}}
\newcommand\stores{{\mathbb{S}}}

\DeclareRobustCommand\cola{$|\cargset(\eargvec)|$}
\DeclareRobustCommand\colb{NK}
\DeclareRobustCommand\colc{NS}
\DeclareRobustCommand\cold{NI}

\newcommand*\blackcircled[1]{\tikz[baseline=(char.base)]{
            \node[shape=circle,draw,inner sep=0.5pt,fill=black,text=white] (char) {#1};}}

\newcommand\mmkernel{\texttt{MM}\xspace}
\newcommand\demoop{\mmkernel}
\newcommand\instantiated[1]{\mmkernel\langle #1\rangle}
\newcommand\kDef[3]{\texttt{#1}(\texttt{#2,}\overrightarrow{#3})\xspace}
\newcommand\kSpecialNm[2]{\texttt{#1}\langle \hlight{#2}\rangle\xspace}
\newcommand\kSpecial[3]{\texttt{#1}\langle \hlight{#3}\rangle(\texttt{#2})\xspace}
\newcommand\kSpecialD[4]{\texttt{#1}\langle \hlight{#3}\rangle(\texttt{#2,}\vec{#4})\xspace}
\renewcommand\vec[1]{\overrightarrow{#1}}
\newcommand\vecC[1]{\hlight{\overrightarrow{#1}}\xspace}

\newcommand{\addr}[2]{#1[#2]}
\newcommand{\compstate}[2]{(#1, #2)}



%%
%% \BibTeX command to typeset BibTeX logo in the docs
\AtBeginDocument{%
  \providecommand\BibTeX{{%
    Bib\TeX}}}

%% Rights management information.  This information is sent to you
%% when you complete the rights form.  These commands have SAMPLE
%% values in them; it is your responsibility as an author to replace
%% the commands and values with those provided to you when you
%% complete the rights form.
\setcopyright{acmlicensed}
\copyrightyear{2018}
\acmYear{2018}
\acmDOI{XXXXXXX.XXXXXXX}
%% These commands are for a PROCEEDINGS abstract or paper.
\acmConference[Conference acronym 'XX]{Make sure to enter the correct
  conference title from your rights confirmation email}{June 03--05,
  2018}{Woodstock, NY}
%%
%%  Uncomment \acmBooktitle if the title of the proceedings is different
%%  from ``Proceedings of ...''!
%%
%%\acmBooktitle{Woodstock '18: ACM Symposium on Neural Gaze Detection,
%%  June 03--05, 2018, Woodstock, NY}
\acmISBN{978-1-4503-XXXX-X/2018/06}


%%
%% Submission ID.
%% Use this when submitting an article to a sponsored event. You'll
%% receive a unique submission ID from the organizers
%% of the event, and this ID should be used as the parameter to this command.
%%\acmSubmissionID{123-A56-BU3}

%%
%% For managing citations, it is recommended to use bibliography
%% files in BibTeX format.
%%
%% You can then either use BibTeX with the ACM-Reference-Format style,
%% or BibLaTeX with the acmnumeric or acmauthoryear sytles, that include
%% support for advanced citation of software artefact from the
%% biblatex-software package, also separately available on CTAN.
%%
%% Look at the sample-*-biblatex.tex files for templates showcasing
%% the biblatex styles.
%%

%%
%% The majority of ACM publications use numbered citations and
%% references.  The command \citestyle{authoryear} switches to the
%% "author year" style.
%%
%% If you are preparing content for an event
%% sponsored by ACM SIGGRAPH, you must use the "author year" style of
%% citations and references.
%% Uncommenting
%% the next command will enable that style.
%%\citestyle{acmauthoryear}


%%
%% end of the preamble, start of the body of the document source.
\begin{document}

%%
%% The "title" command has an optional parameter,
%% allowing the author to define a "short title" to be used in page headers.
\title{The Name of the Title Is Hope}

%%
%% The "author" command and its associated commands are used to define
%% the authors and their affiliations.
%% Of note is the shared affiliation of the first two authors, and the
%% "authornote" and "authornotemark" commands
%% used to denote shared contribution to the research.
\author{Shuo Liu}
% \authornote{Both authors contributed equally to this research.}
\email{zkdliushuo@mail.ustc.edu.cn}
\orcid{1234-5678-9012}
\affiliation{%
  \institution{University of Science and Technoloty of China}
  \city{Hefei}
  \state{Anhui}
  \country{CN}
}

% \author{Lars Th{\o}rv{\"a}ld}
% \affiliation{%
%   \institution{The Th{\o}rv{\"a}ld Group}
%   \city{Hekla}
%   \country{Iceland}}
% \email{larst@affiliation.org}

%%
%% By default, the full list of authors will be used in the page
%% headers. Often, this list is too long, and will overlap
%% other information printed in the page headers. This command allows
%% the author to define a more concise list
%% of authors' names for this purpose.
\renewcommand{\shortauthors}{Shuo et al.}

%%
%% The abstract is a short summary of the work to be presented in the
%% article.
\begin{abstract}
  Abstract here.
\end{abstract}

%%
%% The code below is generated by the tool at http://dl.acm.org/ccs.cfm.
%% Please copy and paste the code instead of the example below.
%%
\begin{CCSXML}
<ccs2012>
 <concept>
  <concept_id>00000000.0000000.0000000</concept_id>
  <concept_desc>Do Not Use This Code, Generate the Correct Terms for Your Paper</concept_desc>
  <concept_significance>500</concept_significance>
 </concept>
 <concept>
  <concept_id>00000000.00000000.00000000</concept_id>
  <concept_desc>Do Not Use This Code, Generate the Correct Terms for Your Paper</concept_desc>
  <concept_significance>300</concept_significance>
 </concept>
 <concept>
  <concept_id>00000000.00000000.00000000</concept_id>
  <concept_desc>Do Not Use This Code, Generate the Correct Terms for Your Paper</concept_desc>
  <concept_significance>100</concept_significance>
 </concept>
 <concept>
  <concept_id>00000000.00000000.00000000</concept_id>
  <concept_desc>Do Not Use This Code, Generate the Correct Terms for Your Paper</concept_desc>
  <concept_significance>100</concept_significance>
 </concept>
</ccs2012>
\end{CCSXML}

\ccsdesc[500]{Do Not Use This Code~Generate the Correct Terms for Your Paper}
% \ccsdesc[300]{Do Not Use This Code~Generate the Correct Terms for Your Paper}
% \ccsdesc{Do Not Use This Code~Generate the Correct Terms for Your Paper}
% \ccsdesc[100]{Do Not Use This Code~Generate the Correct Terms for Your Paper}

%%
%% Keywords. The author(s) should pick words that accurately describe
%% the work being presented. Separate the keywords with commas.
\keywords{Do}

\received{20 February 2007}
\received[revised]{12 March 2009}
\received[accepted]{5 June 2009}

%%
%% This command processes the author and affiliation and title
%% information and builds the first part of the formatted document.
\maketitle

\section{Introduction}\label{sec:intro}

% 随着深度学习的发展,许多大型模型不断涌现并在众多任务中表现出色,如 Llama 3、DeepSeek、Qwen 和 OpenSora等等。
% 为了加速大模型的训练和推理,更好地提供庞大的算力支撑,许多公司开发了各自的领域专用架构(DSA),包括英伟达GPU、华为 Ascend NPU、谷歌 TPU 和寒武纪 MLU 等。
% 作为这些芯片的代表之一,Ascend NPU 目前已经广泛支持了各类有影响力的模型,例如Qwen、DeepSeek系列的模型,Ascend 取得了高性能和低功耗。

With the development of deep learning, many large models have emerged and performed well in various tasks, such as Llama 3, DeepSeek, and Qwen.
To accelerate the training and inference of large models and provide substantial computational support, many companies have developed their own domain-specific architectures (DSA), including NVIDIA GPUs, Huawei Ascend NPUs, Google TPUs, and Cambricon MLUs.
As a representative of these chips, the Ascend NPU has been widely used to support various influential models, such as the Qwen and DeepSeek series, achieving high performance and low power consumption.

% 相较于通用 GPU,Ascend 的特点是额外引入了几种可编程的硬件部件,以高效支撑不同类型的计算:
% 1)计算部件,包括处理控制流、执行指令分发和标量数据计算的 Scalar unit,分别处理向量化和矩阵操作的 Vector 和 Cube Unit。
% 2)多种memory buffer和数据搬运部件:包括面向 Cube Unit 的 L0 A/B/C buffer,以及面向 Vector Unit 的 Unified buffer。这些存储部件之间的数据搬运方式也很灵活,
% 这些硬件部件之间的数据通路我们将在后文中详细介绍,参见xxx.

Compared to general-purpose GPUs, Ascend introduces several additional programmable hardware components to efficiently support different types of computations:
1) Computational components, including the Scalar Unit for handling control flow, instruction dispatch, and scalar data computations, as well as the Vector and Cube Units for processing vectorized and matrix operations, respectively.
2) Various memory buffers and data transfer components: including L0 A/B/C buffers for the Cube Unit and a Unified Buffer for the Vector Unit. The data transfer methods between these storage components are also flexible.

% 上述的设计给开发人员提供了超越 GPU 的可编程性和性能优化潜力,
% 但这也导致 Ascend 核函数的开发者需要更仔细地控制对性能十分重要的并行度以及负载均衡,
% 开发者通常会为算子内的操作定义一组参数,用于控制核函数的每个层级的存储层级上 macro-kernel 的 tile size、迭代顺序、缓冲区数目,以及各个存储层级之间的数据搬运方式。
% 为了支持诸如 Transformer 架构的动态形状模型(输入张量的形状是动态、运行时可变的),上述这组控制参数,连同算子的形状参数、ragged tensor(batch 内 tensor 形状不同)的 offsets 数组、输入输出的数据格式等一起,被实现核函数的形式参数。
% 这些核函数参数需要运行时根据输入形状即时确定再从 Host 传递到 Device,这导致 Scalar Unit 需要实际执行依赖这些参数变量的控制流指令、标量计算并消耗宝贵的通用寄存器(类似于 CPU,每个Davinci Core 核心只有 32 个通用 64 bit 寄存器)存储中间结果。
% 在一类工作负载降低的核函数中,由于 Cube、 Vector 以及数据搬运类的操作次数较少,难以有效掩藏 Scalar Unit 上的任务。
% 尤其是诸如 Flash Decoding (用于 LLM 模型自回归解码阶段执行高效注意力计算)这类工作负载较低、计算逻辑又相对复杂的算子,
% Scalar Unit bound 实际上是这类核函数的一个关键性能瓶颈。

While the above design provides developers with programmability and performance optimization potential beyond GPUs,
it also requires Ascend kernel function developers to carefully control parallelism and load balancing, which are crucial for performance.
Developers typically define a set of parameters for operations within operators to control the tile size of macro-kernels at each level of the memory hierarchy, iteration order, number of buffers, and data transfer methods between different memory levels.
To support dynamic shapes (where input tensor shapes are compile-time unknown and vary at runtime), all these control parameters and shape sizesnm are definded as formal parameters of the implemented kernel functions. 
Besides, to support various Transformer-based model variants with different data formats or 

% , this set of control parameters, along with the operator's shape parameters, offsets arrays for ragged tensors (where tensor shapes differ within a batch), and input/output data formats, are all formal parameters of the implemented kernel functions.

我们统计了 CANNDev、CANN OPS ADV 两个在 Ascend 设备上最广泛使用的算子库的部分算子的各类硬件部件的利用率。
图中纵轴代表算子执行中,一种硬件部件的 active 周期数目。
注意,由于 SIMD 架构的 DSA 多个硬件部件通常并行执行,所有硬件部件的总周期数目实际上大于算子的总执行周期数目。
上图证明 Ascend 的核函数性能对核函数的参数数目敏感,过多的参数可能会导致标量部件成为瓶颈,其他运算和访存部件由于等待标量部件而处于 idle 状态。
进一步地,我们分析发现,固定模型参数(例如注意力头数目、embedding size、最大输入长度等)和硬件配置后,大部分核函数参数都是编译期常量(运行时为不变量且可以在编译期静态求值)。
如表 xxx 所示,这揭示了潜在加速 Ascend 核函数的机会:识别属于编译期常量的核函数参数在编译期替换为常量值,
并通过 DSA native 编译器编译产生针对特定模型和硬件的特化优化的核函数。

据我们所知,当前从 DSA 算子库提取核函数可以被特化的参数的方法主要有两种:
1)对特定的核函数参数执行手工的常量传播;2)基于自动调优 (AutoTuning) 或即时编译确定部分核函数参数。
手工常量传播通常只对少数特定的核函数参数做特化,这种方法预先枚举每个参数的所有可能取值并为每种取值特化生成一个核函数。
这种方式通常只适用于核函数参数较少、手工枚举待特化参数的所有取值不是特别复杂的情况,
由于 GPU 的 SIMT 架构的可编程硬件较少,算子库核函数的参数数目也较少,因而采取了这种方法。
但手工常量特化的不足在于可能会导致严重的代码膨胀问题。
出于泛化性以及减少运行时编译开销的考虑,算子库通常采取 AOT 编译方式,编译前需要适配好各类模型和各种硬件,导致代码和编译产物的基础体积就已经很大了。
因而即便是只枚举部分核函数参数的少量取值,也通常会使代码体积严重增加。
一些 AI Compiler 或者高层张量编程 DSL (Linalg, Triton, etc.) 的算子实现主要通过自动调优 (AutoTuning) 以及即时编译的方式实现自动常量特化核函数的部分参数。
但这种常量特化严重依赖根据运行时形状实际执行 wrapper code 并启用在线常量特化,带来 CPU 上的执行和编译开销。
因而在模型工业部署落地场景中,当前依然以手工算子库为主,AI Compiler 自动生成的算子为辅的方式提供高性能推理服务。

综上原因,我们认为有必要设计一套常量特化框架,实现根据算子的输入形状中的常量维度的取值和算子的 Host code,自动识别出核函数参数中哪些参数是编译时常量,从而实现对核函数的自动常量特化。

针对算子库通常基于 C++ 开发这一事实,本文提出一个基于跨过程常量传播实现的算子自动常量特化方法。
我们的方法可以用于降低算子的推理时延,尤其是降低算子的标量计算开销,也能实现对算子库的代码瘦身,后者在资源受限的端侧设备以及无服务计算上更重要。


在编译时,我们基于深度学习框架(例如 PyTorch Dynamo),运行动态形状模型,捕获得到每个算子的每个输入输出的 symbolic shape 以及算子属性(例如注意力头数目、embedding size)的常量值。
此外,我们手动为硬件构造了一份配置文件,描述了 Host Code 可能会查询的硬件基本信息,例如计算核心数目以及 L1/L2 Cache 的大小等信息。
对于整个算子库,我们将其中的 Host 程序降低到 LLVM IR,我们为每个算子的 Host Code 的入口函数根据前面获取的输入输出形状和算子属性信息,构造一份常量特化的上下文。
基于算子程序 LLVM IR 和常量特化上下文,我们提出一种基于抽象解释的常量传播框架,称作 \mysys 用于推导出核函数的编译期常量的参数值。
动态形状检查在算子程序中普遍存在,针对变量维度及其表达式的条件检查语句会造成多条程序路径,为此我们提出一种识别不可达路径的方法来避免分析精度的降低。
此外,针对 C++ 算子程序中普遍使用的多态类型以及指针,为提升分析精度,我们在抽象解释中引入了指针指向的传播。
此外,当遇到不确定的指针指向时,不同于传统的方法为了保证分析的正确需要将所有可能指向的内存区域都设置为 Unknown,这可能会导致我们的分析对象被污染,我们采取了相当激进的实现:忽略不确定的指针指向,不更新任何内存,我们将正确性的检查留给运行时的正确性检测。
通过上述常量传播,我们可以分析得出核函数的部分参数取值,并将这些参数替换为对应的常量,再对核函数进行常量相关的优化,得到特化的核函数程序。
此外,我们用特化的参数为特化核函数安装一个运行时正确性检测的 guard,确保特化核函数的正确性。
在运行时,对于给定的运行时形状,我们首先运行一遍 guard,若满足约束则调用特化核函数;否则,调用原版非特化的核函数。


\section{Background}
\label{sec:overview}
\subsection{Ascend Architecture and Challenges}

\begin{figure}
  \includegraphics[width=0.7\textwidth]{figures/motivation/ascend-data-path.png}
  \caption{Overview of Ascend Architecture. The figure illustrates the key components of the Ascend architecture.}
  \label{fig:ascend_architecture}
\end{figure}


The Ascend architecture, as illustrated in \autoref{fig:ascend_architecture}, is designed to efficiently handle diverse deep learning workloads by incorporating specialized hardware components. Key components include:
\begin{itemize}
    \item \textbf{Computation Units:} The architecture features heterogeneous computation units: \textbf{Scalar Units} for control flow processing, instruction dispatching, and scalar arithmetic; \textbf{Vector Units} for SIMD (Single Instruction, Multiple Data) operations; and \textbf{Cube Units} for high-throughput matrix multiplications. These units collaborate to execute complex neural network operators efficiently.
    
    \item \textbf{Memory Hierarchy:} Ascend employs a multi-tiered memory hierarchy with specialized buffers. The L0 A/B/C buffers are dedicated to Cube Units, while the Unified Buffer serves Vector Units. This design facilitates high-bandwidth data movement between different storage tiers, optimizing memory access patterns.
    
    \item \textbf{Data Movement Mechanisms:} The architecture includes flexible Direct Memory Access (DMA) mechanisms that enable efficient data transfer between computation units and memory buffers. This is crucial for maintaining high pipeline utilization and minimizing latency.
\end{itemize}

While these components provide a highly programmable and high-performance platform, they introduce significant challenges in kernel optimization compared to general-purpose GPUs. 
Ascend developers must meticulously manage parallelism and load balancing through a vast set of parameters controlling tile sizes, iteration orders, buffer counts, and data movement patterns across the memory hierarchy. 
For dynamic-shape models (e.g., Transformers with runtime-variable sequence lengths), these scheduling parameters, along with tensor shapes and offsets, are passed as formal arguments to kernel functions. 

At the start of kernel execution, these parameters must be loaded from global memory into registers or the Unified Buffer. The Scalar Unit then executes control flow instructions and scalar computations dependent on these variables. 
Since the Scalar Unit typically has a limited number of general-purpose registers, processing a large number of dynamic parameters triggers frequent load/store operations and register spilling. 
This results in a \textbf{Scalar Bottleneck}, where the latency of the Scalar Unit limits the overall throughput. 
In workloads with low computational intensity but high scheduling complexity, such as Flash Decoding, this scalar overhead consumes a significant portion of the execution time.

We observe that while input shapes vary dynamically, many kernel parameters effectively become \textit{compile-time constants} once model configurations (e.g., attention heads, embedding sizes) and hardware specifications are fixed. 
This reveals an opportunity to accelerate Ascend kernels by identifying these constants and generating specialized kernels via the native compiler. 
However, identifying these constants automatically from C++ based operator libraries presents unique challenges, discussed next.

\subsection{Current Kernel Specialization Methods}

There are two primary methods are currently employed for specializing kernel parameters in other DSA operator libraries(e.g., GPU): Ahead-of-Time (AoT) compilation and Just-in-Time (JIT) compilation.
\autoref{lst:compare_aot_jit_without_graph} demonstrates specializing the tile size parameter \kw{T} in a simple addition kernel using both methods.

\begin{figure}[htbp]
\begin{minipage}{0.39\textwidth}
\begin{lstlisting}[basicstyle=\ttfamily\footnotesize, caption={Templated kernel for kernel \kw{add}}, captionpos=b, frame=single]
template <int T>
void kernel(float* A, float* B, float* C, int s) {
  for (int i = 0; i < s; i += T) {
    for (int j = 0; j < T; ++j) {
      int idx = i + j;
      if (idx < s)
        C[idx] = A[idx] + B[idx];
    }
  }
}
\end{lstlisting}
\end{minipage}
\hfill
\begin{minipage}{0.56\textwidth}
\begin{lstlisting}[firstnumber=11, basicstyle=\ttfamily\footnotesize, caption={AoT and JIT implementations of operator \kw{add}}, captionpos=b, frame=single]
status aot_add(Tensor& A, Tensor& B, Tensor& C) {
  int t = 0; tiling(A.size(0), t);
  switch (t) {
    case 16: kernel<16>(..., A.size(0)); break;
    case 32: kernel<32>(..., A.size(0)); break;
}}
status jit_add(Tensor& A, Tensor& B, Tensor& C) {
  int t = 0; tiling(A.size(0), t);
  auto kernel_ = run_compile(kernel, t);
  kernel_(A.ptr, B.ptr, C.ptr, A.size(0));
}
\end{lstlisting}
\end{minipage}
\caption{Comparison of AoT compilation and JIT compilation for kernel specialization of add operator.\scnote{\kw{aot\_add} returns status but \kw{jit\_add} returns void}}
\label{lst:compare_aot_jit_without_graph}
\end{figure}

\textbf{AoT Compilation:} As shown in \kw{aot\_add}, developers manually precompile kernels for a predefined set of parameter values, enumerated using a \kw{switch-case} statement. 
This approach enjoys little runtime overhead but is severely limited by the \textit{code bloat} problem. 
With operator libraries often containing dozens of parameters, exhaustively enumerating all possible combinations becomes infeasible.
As a result, AoT methods can only specialize a small subset of parameters.
Moreover, for a specific model deployment, most parameters are fixed, meaning only a tiny fraction of these precompiled kernels are ever utilized, while the rest remain redundant.

\textbf{JIT Compilation:} As shown in \kw{jit\_add}, the parameter value \kw{t} is determined at runtime, and a specialized kernel \kw{kernel\_} is compiled for the specific \kw{t}. 
This approach provides maximum flexibility but incurs significant overhead due to runtime compilation and storage requirements. 
For latency-sensitive inference scenarios, this overhead is often prohibitive, even when caching mechanisms are employed to reuse compilation results.

The key observation is that once model hyperparameters and hardware configurations are fixed, most kernel parameters can be treated as compile-time constants. 
Existing AoT and JIT methods attempt to specialize a small subset of these parameters but fail to fully exploit this insight due to their inherent limitations. 
To overcome these challenges, it is essential to design a constant specialization framework capable of automatically identifying compile-time constants based on the specific model context and generating specialized kernels offline, thereby avoiding both code bloat and runtime overhead.

\subsection{Existing Constant Propagation Techniques}

To achieve automatic specialization, one must rely on static program analysis to deduce constant values. Existing constant propagation techniques can be broadly categorized into two methodologies:

\begin{itemize}
    \item \textbf{Data Flow Analysis (DFA):} 
    DFA is a classic algorithm grounded in the theory of Abstract Interpretation. It models the program as a Control Flow Graph (CFG) and propagates abstract values (e.g., lattice elements like \textit{Top}, \textit{Constant}, \textit{Bottom}) iteratively until a fixed point is reached. 
    DFA offers a good balance between precision and efficiency. 
    However, standard DFA is typically \textbf{path-insensitive}. When control flows merge (e.g., after an \kw{if-else} block), DFA merges the abstract states from all incoming paths using a \kw{join} operator. This often results in a loss of precision, degrading a constant value to \textit{Unknown} (Top) if branches disagree.

    \item \textbf{Symbolic Execution (SE):} 
    SE explores individual program paths, maintaining the program state as symbolic expressions and tracking path constraints. 
    It is \textbf{path-sensitive}, distinguishing values across different execution paths, which offers higher precision than DFA. 
    However, SE suffers from the \textbf{path explosion} problem. As the number of branches increases, the number of paths grows exponentially, making SE computationally prohibitive for analyzing large-scale C++ operator libraries.
\end{itemize}

These techniques are mature for single-language analysis (e.g., analyzing pure C++ or LLVM IR). However, applying them to Deep Learning (DL) frameworks introduces a novel \textbf{Cross-Language Analysis} challenge. 
DL frameworks typically use Python for the frontend and C++ for the backend operator library. 
There exists a significant "semantic gap" between the Python runtime (where constant shape values are known) and the C++ static analysis environment (where the kernel code resides).

To the best of our knowledge, existing cross-language analysis tools primarily focus on type checking, security, or interface binding generation. 
There is a lack of work on \textbf{propagating concrete runtime values from a high-level dynamic language (Python) to the low-level binary layout of a static language (C++)} to enable bit-level constant propagation for compiler optimizations.
Specifically, two main obstacles hinder the direct application of existing DFA or SE:

\begin{enumerate}
    \item \textbf{Semantic opaque across language boundaries:} 
    The constant information (e.g., tensor shapes) is available in the Python runtime context. However, when analyzing the C++ operator binary (or IR), this information is lost. The analyzer sees only opaque pointers to C++ objects (like \kw{torch::Tensor}) without knowing the memory layout or the specific values stored within. This makes it impossible to initialize the analysis context correctly using only C++ static analysis tools.
    
    \item \textbf{Precision loss from dynamic shape checks:} 
    Operator libraries contain extensive defensive checks for dynamic shapes (e.g., \kw{if (shape \% 32 != 0) return error;}). 
    Since some dimensions remain symbolic during partial evaluation, a path-insensitive DFA cannot determine which branch is taken. It conservatively merges the "error path" (where values are undefined) with the "valid path," causing constant values to degrade to \textit{Unknown}. While SE could theoretically handle this, the path explosion from numerous checks makes it impractical.
\end{enumerate}

The above challenges motivate us to develop \mysys, as described in \autoref{sec:design}, which addresses the cross-language gap via context reconstruction and enhances precision through pattern-based pruning.

%sxy% 第二章Background和第三章challenge章节,看起来内容非常充实并且有很多例子。但是结构组织上让我有点困惑:例如第二章Background中似乎也在分析challenge;而第三章challenge中还介绍了本文克服challenge的Approach。
%sxy% 我建议更清晰的组织章节:在第二章Background中只介绍背景知识,在第三章challenge中集中分析挑战,在后续介绍Approach的章节中再介绍本文是如何克服challenge的。另一种改法是将Background和challenge两个章节合并为一个章节,不过这个章节可能会过长。
% C++ 算子程序中普遍使用库 API 获取算子属性,从源码分析的角度,这些库 API 函数是不可见的黑盒函数,需要首先解决常量信息从哪里来的的问题。
% 此外,动态形状检查在算子程序中普遍存在,针对变量维度及其表达式的条件检查语句会造成多条程序路径,传统的路径不敏感的分析会导致常量信息的大量丢失,而路径敏感分析会造成昂贵的编译开销。

% 通过上述常量传播,我们可以分析得出核函数的部分参数取值,并将这些参数替换为对应的常量,再对核函数进行常量相关的优化,得到特化的核函数程序。
% 在运行时,对于给定的运行时形状,我们首先运行一遍 guard,若满足约束则调用特化核函数;否则,调用原版非特化的核函数。

\section{Challenges in Constant Propagation and Insights}

Our targeted operator programs involve some challenging patterns that hinder the effectiveness of existing constant propagation techniques.
In this section, we discuss the impact of these patterns on constant propagation and our key insights.

\subsection{Propagating constant informations through Python/C++ Bindings}

Deep learning models are typically defined in high-level frameworks like PyTorch using Python APIs. The model is organized as a computation graph where each node represents an operator. The input shapes and attributes of these operators, which contain the runtime invariant values we wish to propagate, are passed to the underlying C++ implementations through Python bindings.
However, recovering the C++ side calling context from the Python API invocation presents a significant challenge known as the \textbf{Cross-Language Semantic Gap}.

Specifically, to perform static constant propagation on the C++ operator programs, the analyzer must identify which memory load instructions correspond to the constant values known by the framework.
This is difficult because Python objects are high-level wrappers, while C++ analysis requires precise, bit-level memory states.
We illustrate this challenge with an example of the \kw{add} operator in \autoref{fig:semantic_gap}.

\begin{figure}[htbp]
\centering
% Python View
\begin{lstlisting}[language=Python, caption={Python API Invocation of the operator \kw{add}}, label={lst:python_view}, frame=tlrb]
a, b, c= torch.randn(4), torch.randn(s0), torch.randn(4)
torch.add(a, b, c) # Invokes C++ implementation
\end{lstlisting}
\end{figure}

\begin{figure}[htbp]
\begin{minipage}{0.5\textwidth}
\begin{lstlisting}[language=C++, caption={C++ Implementation of the operator \kw{add}}, label={lst:cpp_view}, frame=tlrb]
struct TensorImpl {
  ... // Other metadata
  SmallVector<int64_t, 5> sizes_;
};
void add(Tensor& A, Tensor& B, Tensor& C) {
  int s = A.impl_->sizes_[0]; 
  ... // Other code
}
\end{lstlisting}
\end{minipage}
\hfill
% LLVM IR View
\begin{minipage}{0.47\textwidth}
\begin{lstlisting}[language=llvm, caption={LLVM IR of the operator \kw{add}}, label={lst:llvm_view}, frame=tlrb]
%impl_ptr = load %struct.TensorImpl*, %struct.Tensor* %a_ptr
%sizes_base = getelementptr i8, i8* %impl_ptr, i64 24
%dim0_addr = bitcast i8* %sizes_base to i64*
; We should replace the following load instruction with constant 4
%val = load i64, i64* %dim0_addr
\end{lstlisting}
\end{minipage}
\caption{The Cross-Language Semantic Gap. Framework-level constants (e.g., shape dimension `4`) are known in Python but physically invisible to the C++ static analyzer unless the memory context is reconstructed at the bit level.}
\label{fig:semantic_gap}
\end{figure}

As illustrated in \autoref{fig:semantic_gap}, the framework sees the concrete shape \kw{[4]}, yet the compiled C++ code degenerates into LLVM IR that performs raw pointer arithmetic and byte loads.
No existing cross-language tooling exposes these constants in Python-side codes to the C++ analyzer. 
Enabling constant propagation therefore requires:
\begin{enumerate}
    \item \textbf{Reconstruct Bit-Level Context.} We must synthesize an initial memory image where the bytes at the offset corresponding to \kw{TensorImpl::sizes\_[0]} contain the binary encoding of \kw{4}.
    \item \textbf{Identify Variable Fields.} We must also recognize that \kw{sizes\_[0]} of tensor \kw{B} represents the symbolic dimension \kw{s0} and explicitly mark those bytes as \kw{Unknown} (Top) to keep the analysis sound.
\end{enumerate}

To bridge this gap, we propose a method to reconstruct a precise C++ calling context.
Conceptually, this context serves as a mapping that associates the memory locations of input parameters with their abstract values, either specific constants (e.g., shape dimensions) or \kw{Unknown} (Top).
To do so, we first analyze the AST of the operator library to recover the physical memory layout (i.e., bytes offsets of fields) of the parameter data structures.
Second, we predefine the mapping rules to map the known constant values from the Python frontend into their corresponding offsets for each parameter, while initializing all other variable fields by mapping to \kw{Unknown}.
The feasibility of this approach relies on two key observations: (1) C++ operator programs utilize a limited set of parameter types (e.g., \kw{Tensor}, \kw{int}, \kw{float}) and , and (2) constant propagation primarily depends on shape dimensions and operator attributes. 
Consequently, this reconstructed context provides the necessary initial state for the subsequent static analysis.

\subsection{Dynamic Shape Checks leading to Early Return Paths}

Dynamic shape operators need to employ shape checks to detect invalid inputs for robustness due to several reasons:
1) Some high-performance algorithms only support specific shapes or data layouts. 
2) Inputs may exceed resource limits for the target hardware.
Consequently, runtime shape checks are prevalent in operator implementations, especially for NPU operators, which must satisfy hardware constraints such as mandatory memory alignment, tile volume should not exceed certain memory buffer size, etc.
These checks often locate in the beginning of tiling functions, and the error reporting path usually returns an error status code and bypasses the rest of the function, called \kw{early return path}.

\begin{figure}[H]
    \centering
    \hfill
    \begin{minipage}{0.5\textwidth}
    \begin{lstlisting}[language=C++, caption={Early return path due to dynamic shape check.}, label={lst:early_return_code}, frame=single]
const int SUCCESS = 0, ERROR = -1;
int check(int s) {
    if(s % 32 != 0)
        return ERROR; // fails
    return SUCCESS;
}
int tiling(int s, int& t) {
    int ret = check(s);
    if (ret != SUCCESS) // check
        return ret; // check fails
    t = 32;
    return SUCCESS;
}
    \end{lstlisting}
    \end{minipage}
\hfill
    \begin{minipage}{0.4\textwidth}
    \includegraphics[width=\textwidth]{figures/motivation/dynamic-shape-check-demo.png}
    \caption{Data flow and control flow of the function \kw{tiling} in \autoref{lst:early_return_code}.}
    \label{fig:dynamic_shape_check_demo}
    \end{minipage}
\end{figure}

As shown in \autoref{lst:early_return_code}, the function \kw{tiling} contains a dynamic shape check in the called function \kw{check}.
If the input shape parameter \kw{s} is not aligned to 32 (i.e., \kw{s} \% 32 $\neq$ 0), the function \kw{check} returns an error code, leading to an early return in \kw{tiling} that bypasses the assignment of the output parameter \kw{t}.
% [Modification: Explicitly stating the initial value assumption]
Assuming the initial value of \kw{t} is 0 passed from the caller, the actual trouble arises when we try to deduce the constant value of \kw{t} using constant propagation at block 3.
As shown in \autoref{fig:dynamic_shape_check_demo}, there are two possible paths to reach block 3:
1) The valid path is from block 2 where \kw{s} is a multiple of 32, and \kw{t} is assigned the constant value 32.
2) The early return path is from block 1 where \kw{s} is not aligned, and \kw{t} remains its initial value 0.
Consequently, at block 3, the value of \kw{t} is a merge of 0 and 32, resulting in a non-constant value (often denoted as \textit{Top} in lattice theory).
Although symbolic execution techniques may explore paths and deduce values under symbolic guards, they suffer from path explosion and are computationally expensive, making them impractical for our task.

% [Modification: Correcting "Unreachable" to "Irrelevant for valid inputs/Data flow analysis"]
The key insight is that the control flow edge from Block 1 to 3 represents an error handling routine triggered by invalid inputs.
Under the assumption that the optimized kernel is intended for valid workloads, and following the API contract that output parameters are undefined upon failure, the data flow values from this early return path can be safely ignored.
By excluding this path from data-flow analysis, we can deduce that \kw{t} is consistently 32 at block 3 for all successful executions.

Determining which control flow edges belong to error handling paths through static program analysis is non-trivial in general, as distinguishing error reporting paths from normal conditional branches relies on semantic information typically absent in low-level intermediate representations. 
Fortunately, dynamic shape operator programs follow certain patterns that can be exploited.
First, these programs commonly use compile-time known status codes to indicate success or failure of dynamic checks.
For example, the SUCCESS code is always 0, and the ERROR code is always -1 in all the built-in operator libraries. 
Second, the usage of these status codes follows a consistent pattern:
1) A function performing dynamic shape checks will return a status code to indicate success or failure to its caller.
2) The caller checks the callee's status; if the comparison indicates failure, the return value or ERROR code will be returned immediately.

Based on the above observations, we identify early return paths by leveraging the explicit return of error status codes.
In general, we identify paths that consistently lead to a function return of an ERROR code. 
Once identified, we mark these paths as Early-Return and modify the constant propagation pass to selectively merge incoming values only from non-error paths.
This strategy ensures that the constant propagation is not polluted by the uninitialized or default values from error handling logic, while preserving correctness for valid program executions.

% Based on the above observations, we have designed an early return path identification algorithm to identify unreachable early return paths caused by dynamic shape checks.
% In general, for a given function that performs dynamic shape checks, we first extract all possible check failure branches, then we analyze if on all the paths from a check failure branch to the function's exit block, the return codes are all ERRORs.
% If so, we mark these paths as early return paths that can be ignored during constant propagation.

% Due to the coorectness requirement, we only ignore early return paths when we are sure that they are unreachable given valid inputs.
% This is a conservative strategy that may miss some optimization opportunities, but it guarantees the correctness of the optimized program.



% \subsection{The Challenge of Pointer Uncertainty}

% Constant propagation (CP) is critically hindered by \textit{unknown pointers}, where a \texttt{store} operation may poison all potentially-aliased memory locations, losing many constant values.
% There are 

% While existing alias analysis (e.g., TBAA in LLVM) can mitigate this by restricting side-effects to a specific field, this is insufficient for two patterns prevalent in operator host code. 
% These patterns create pointer uncertainty so severe that it defeats even advanced conservative analysis.

% \begin{figure}[h]
% \centering
% \begin{minipage}{.48\textwidth}
% \lstset{style=cppsimple, title=(a) External Function}
% \begin{lstlisting}
% extern int getEnvData();
% void my_kernel(int param);

% void host_func_1(...) {
%   int val = getEnvData();
%   my_kernel(val); 
% }
% \end{lstlisting}
% \end{minipage}
% \hfill
% \begin{minipage}{.48\textwidth}
% \lstset{style=cppsimple, title=(b) Dynamic STL Container}
% \begin{lstlisting}
% #include <vector>
% void my_kernel(int param);

% void host_func_2(int shape) {
%   std::vector<int> v;
%   v.push_back(10);
%   if (shape > 0)
%     v.push_back(shape);
%   my_kernel(v[0]);
% }
% \end{lstlisting}
% \end{minipage}
% \caption{Patterns creating unanalyzable uncertainty. (a) The opaque call \texttt{getEnvData()} forces \texttt{val} to become \texttt{UNKNOWN}. (b) The dynamic \texttt{push\_back(shape)} implies a potential \texttt{realloc}, forcing a conservative analysis to poison the known constant \texttt{v[0]}.}
% \label{fig:opaque_code}
% \end{figure}

% The first challenge is \textbf{external "black-box" functions} (Fig. \ref{fig:opaque_code}a). Operator code frequently links against opaque runtime APIs whose IR is unavailable. Traditional inter-procedural analysis (IPA) attempts to solve this via Link-Time Optimization (LTO) or manually-crafted function summaries \cite{SomeLTOAnalysisPaper, SomeFunctionSummaryPaper}. These methods are inapplicable in our domain: LTO is impossible against pre-compiled runtime APIs, and summaries for proprietary vendor libraries are unavailable. Thus, a sound analysis must conservatively assume arbitrary side effects (e.g., data escape), invalidating all constants.

% The second challenge is \textbf{dynamic STL containers} (Fig. \ref{fig:opaque_code}b). Although the source code for STL is visible, its behavior is data-dependent. To precisely analyze the heap, advanced compilers may employ complex, high-cost techniques like shape analysis \cite{SomeShapeAnalysisPaper} or symbolic execution. These methods are computationally prohibitive for our domain and still default to conservative assumptions when faced with dynamic, shape-dependent control flow. The standard conservative approach must assume that a \texttt{push\_back} implies a potential \texttt{realloc}, invalidating all element pointers and forcing the analyzer to poison all existing constants (like \texttt{v[0]}) to \texttt{UNKNOWN}.

% To overcome the failure of conservative methods, we leverage two domain-specific insights:
% \begin{enumerate}
%     \item \textbf{Insight 1 (External Funcs):} Manual analysis reveals external calls are overwhelmingly benign (e.g., \texttt{readonly}, \texttt{nocapture}), typically fetching read-only data.
%     \item \textbf{Insight 2 (STL Containers):} Container element counts, while dynamic, are practically bounded (e.g., < 100) in operator code.
% \end{enumerate}

% These insights motivate our "speculative" framework. We analyze code *as if* these problems do not exist: external functions are assumed benign, and containers are modeled via a \textbf{bounded static abstraction} (e.g., \texttt{std::vector} as a \texttt{T[256]} array). Our analysis is \textit{not} conservative; if the bound is exceeded, it continues to propagate constants optimistically. We delegate correctness entirely to a runtime \textbf{guard} that verifies constants before executing a specialized kernel, unifying both challenges under a "speculate-then-verify" model.

\section{System Design of \mysys}
\label{sec:design}

We propose \mysys, a highly accurate constant propagation framework that infers constant parameter values of kernels from deep learning frameworks, specifically PyTorch.
\mysys takes the ATen IR of the computation graph and the LLVM program of the operator as input.
We first generate the calling context for each operator invocation.
Starting from the entry function of the operator, \mysys performs context-sensitive interprocedural constant propagation, propagating constants through each basic block.
At the end of the analysis, the derived constant values for the concrete parameters of the kernel function are used to perform kernel specialization and optimization.

\begin{table}[htbp]
  \centering
  \small
  \renewcommand{\arraystretch}{1.1}
  \setlength{\tabcolsep}{6pt} % 标准表格稍微宽一点也没关系
  \caption{Abstract domains and definitions in \mysys.}
  \label{fig:abstract-domain}
  \begin{tabular}{ll}
    \hline
    \textbf{Domain} & \textbf{Notation} \\
    \hline
    Top-level Variable & $v \in \mathcal{V}$ \\
    Address-taken Var & $o \in \mathcal{O}$ \\
    Constant & $c := \text{Int} \mid \text{Float} \mid \text{FuncPtr}$ \\
    Pointer & $p := \langle \mathcal{B}, \delta \rangle$ \\
    Spec Value & $sv := c \mid p \mid \top \mid \bot$ \\
    Environment & $\mathbb{E} := \mathcal{V} \mapsto \mathcal{SV}$ \\
    Store & $\mathbb{S} := \mathcal{O} \mapsto \mathcal{SV}$ \\
    Calling Context & $\mathbb{C} := (\mathbb{E}, \mathbb{S})$ \\
    \hline
  \end{tabular}
\end{table}

\subsection{Definitions}\label{sec:design:definitions}

To formally describe our analysis, we define the abstract domains and symbols used in \mysys, as summarized in \autoref{fig:abstract-domain}.

\paragraph{Variables and Values.}
We classify variables into two categories: \kw{top-level variables} ($v \in \mathcal{V}$) and \kw{address-taken variables} ($o \in \mathcal{O}$).
A \kw{top-level variable} $v$ corresponds to an SSA (Static Single Assignment) value in LLVM IR, representing a virtual register defined by an instruction or a function argument.
An \kw{address-taken variable} $o$ represents an abstract memory object, such as a stack allocation (\texttt{alloca}) or a global variable, whose address can be manipulated.

The analysis state of these variables is represented by a \kw{specialization value} ($sv \in \mathcal{SV}$).
An $sv$ represents the abstract lattice value of a particular instruction or memory location.
The domain $\mathcal{SV}$ forms a lattice containing the following elements:
\begin{itemize}
    \item \textbf{Top ($\top$):} Represents an undefined or uninitialized state (identity element for the merge operation).
    \item \textbf{Bottom ($\bot$):} Represents an unknown or overdefined state (result of conflicting values).
    \item \textbf{Constant ($c$):} A concrete value, which can be an integer scalar, a floating-point number, or a function pointer.
    \item \textbf{Abstract Pointer ($p$):} Represented as a pair $\langle \mathcal{B}, \delta \rangle$.
    Here, $\mathcal{B} \subseteq \mathcal{O}$ is a set of potential base memory objects pointed to.
    $\delta$ represents the offset from the base, which can be a concrete integer or $\bot$ (if the offset is unknown).
\end{itemize}

\paragraph{Context and State.}
We define the program state using an \kw{Environment} $\mathbb{E}$ and a \kw{Memory Store} $\mathbb{S}$.
The Environment $\mathbb{E}: \mathcal{V} \mapsto \mathcal{SV}$ maps top-level variables to their specialization values, tracking the states of virtual registers.
The Memory Store $\mathbb{S}: \mathcal{O} \mapsto \mathcal{SV}$ maps address-taken variables (memory objects) to their stored values, modeling the heap and stack contents.
Consequently, a \kw{Calling Context} $\mathbb{C}$ is defined as a tuple $(\mathbb{E}_{entry}, \mathbb{S}_{entry})$, representing the initial state upon entering a function.
Specifically, $\mathbb{C}$ captures the specialized arguments: scalar parameters are mapped within $\mathbb{E}_{entry}$, while memory contents accessible via pointer parameters are initialized in $\mathbb{S}_{entry}$.

\subsection{Calling Context Generation for C++ Operators}\label{sec:design:context_construction}

\begin{figure}[htbp]
  \centering
  \includegraphics[width=\linewidth]{figures/sys-design/context-generation-through-pybind.png}
  \caption{Overview of context generation through PyBind in \mysys. This figure illustrates the workflow for reconstructing the bit-level initial state of C++ operators. In the showed memory layout, dashed arrows denote pointer/reference relationships between objects, while solid arrows denote structural field-containment (i.e., one aggregate type embedding another as a member field). Elements with the same background shading denote the same underlying function or class across different stages. The suffix "\_1" of \texttt{add\_1} denotes the first \texttt{add} operator in the model and distinguishes different occurrences of the same operator kind.}
  \Description{A diagram showing the workflow for reconstructing the bit-level initial state of C++ operators, including static analysis, trace analysis, and context synthesis. In the Memory layout subfigure, dashed arrows denote pointer/reference relationships between objects, while solid arrows denote structural field-containment (one aggregate embedding another as a member). Elements with the same background shading denote the same underlying function or class across different stages.}
  \label{fig:context_construction}
\end{figure}

\autoref{fig:context_construction} illustrates the overall workflow for reconstructing the bit-level initial state ($\mathbb{C}_{init}$) of C++ operators. 
The workflow bridges the gap between high-level Python execution and low-level C++ memory states through a hybrid approach.
As shown in \autoref{fig:context_construction}, the context generation module of \mysys consists of three core components highlighted with a \textbf{green background}, namely \textbf{Static Analysis}, \textbf{Trace Analysis}, and \textbf{Context Synthesis}. 
In the figure, elements with identical background shading refer to the same underlying function or class, consistently tracked across the different stages of the pipeline.
The process is divided into an offline preparation phase and an online synthesis phase.

\paragraph{Offline phase: static analysis of binding APIs and memory layout}

The offline phase (top of \autoref{fig:context_construction}) builds a comprehensive knowledge base of the codes of C++ operator library.
We employ a \textbf{static analysis} component (implemented based on Clang) to scan the library source code.
This module produces two key outputs:
\begin{itemize}
  \item \textbf{C++ Operator Metadata}: It locates Python–C/C++ bindings (e.g., mapping Python’s \texttt{add} to its C++ implementation) and records the corresponding function signatures.
  \item \textbf{Memory Layout of C++ Types}: It recursively parses the AST to recover concrete memory layouts. 
  For example, for the \texttt{Tensor} type, it infers a pointer to \texttt{TensorImpl}, which in turn contains nested structures such as \texttt{sizes} (within \texttt{std::vector<int>}).
\end{itemize}

\paragraph{Online phase: trace analysis and context synthesis}

The online phase (bottom of \autoref{fig:context_construction}) captures runtime information from the DL model to instantiate the context.

\textbf{Trace analysis.}
During model execution (e.g., \texttt{torch.add(a, b, ...)}), we invoke the \textbf{trace analysis} component (built on top of the tracing mechanism of compilers such as Inductor) to inspect the execution graph.
Unlike standard execution, this stage performs symbolic execution to preserve information about symbolic shapes.
For example, as illustrated in \autoref{fig:context_construction}, the input script marks tensor \texttt{b} as dynamic via \texttt{mark\_dynamic}.
Consequently, the framework’s runtime knows that the first dimension of \texttt{a} is a concrete constant \texttt{4} (from the Python layer), whereas the corresponding dimension of \texttt{b} is symbolic (unknown at compile time, denoted as ?) as indicated by the framework’s symbolic engine.

\textbf{Context synthesis.}
Finally, the \textbf{context synthesis} module combines the offline memory-layout information with the online argument values and produces the initial calling context $\mathbb{C}_{init} = (\mathbb{E}, \mathbb{S})$ for each C++ operator (e.g., \texttt{add\_1} in \autoref{fig:context_construction}).
Given the traced Python-level arguments, it applies a set of predefined \textbf{instantiation rules} that describe how to allocate abstract objects and initialize their fields in the environment $\mathbb{E}$ and store $\mathbb{S}$.
During this process, the synthesizer reconstructs pointer chains between abstract objects (e.g., $o_{t} \rightarrow o_{impl} \rightarrow o_{data}$) and fills the corresponding abstract memory locations.

The synthesis explicitly preserves the uncertainty observed in the trace.
For each field, if the traced value is a concrete constant (such as a static dimension \texttt{4}), the corresponding address in $\mathbb{S}$ is initialized with a constant specialization value.
If the value is symbolic or otherwise unknown at compile time, the rule writes $\top$ to that address, representing an “unknown but live” specialization value in our abstract domain.
This guarantees that the constructed $\mathbb{C}_{init}$ is sound for subsequent static analysis.

As a concrete example, consider the instantiation rule \textsc{[Tensor]} for a tensor argument.
Given a top-level variable $v$, a pointer to the tensor, and its corresponding traced shape list $L = [d_0, \dots, d_{n-1}]$ (where each $d_k$ may be a constant or a symbolic dimension), the rule allocates fresh abstract objects $o_t$, $o_{impl}$, and $o_{data}$, and updates the environment and store as specified in the inference rule.
Here, $o_t$ models the \texttt{Tensor} object, $o_{impl}$ models the internal \texttt{TensorImpl}, and $o_{data}$ models the contiguous buffer storing the tensor sizes (e.g., the internal array of \texttt{std::vector<int>}).
The offset $\delta_{sz}$ denotes the position of the \texttt{sizes} field within \texttt{TensorImpl}, and $w$ is the stride (in bytes) of one size element.
The helper function $\alpha$ maps a concrete dimension $d_k$ to a constant specialization value in $\mathcal{SV}$, and maps a symbolic $d_k$ to $\top$, capturing that the value must be treated as unknown by the analysis.
This rule-based instantiation mechanism can generalize to other framework-specific types and provide the foundation for constructing precise and sound C++ calling contexts in \mysys.

\begin{equation}
  \textsc{[Tensor]} \quad
  \frac{
    \begin{aligned}
      o_{t}&, o_{impl}, o_{data}~ \text{ are fresh} \\
      L &= [d_0, \dots, d_{n-1}] \\
      \mathbb{E}'& = \mathbb{E}[v \mapsto \addr{o_{t}}{\mathbf{0}}]
    \end{aligned}
    \qquad
    % Part 3: Store Update (Right)
    \mathbb{S}' = \mathbb{S} \left[ 
      \begin{aligned}
        \addr{o_{t}}{\mathbf{0}} & \mapsto \addr{o_{impl}}{\mathbf{0}}, \\
        \addr{o_{impl}}{\delta_{sz}} & \mapsto \addr{o_{data}}{\mathbf{0}}, \\
        \addr{o_{impl}}{\delta_{sz} + \delta_{cap}} & \mapsto n, \\
        \forall k \in [0, n).\; \addr{o_{data}}{k \cdot w} & \mapsto \alpha(d_k)
      \end{aligned}
    \right]
  }{
    % Conclusion
    \langle \textsc{Tensor}(v, L), \compstate{\mathbb{E}}{\mathbb{S}} \rangle \longrightarrow \compstate{\mathbb{E}'}{\mathbb{S}'}
  }
\end{equation}

\textbf{Why predefine rules for C++ calling-context synthesis?}
Although trace analysis preserves rich symbolic information for Python-level arguments (e.g., symbolic tensor dimensions), this information is disconnected from the concrete memory layout used by C++ operator implementations.
In particular, there is no automatically available mapping from a framework-level symbolic variable to its exact byte offset and container object in the C++ layout (e.g., an element of \texttt{TensorImpl::sizes\_} inside a nested \texttt{std::vector<int>}).
Reconstructing such a mapping would require precisely tracking how values flow from the Python runtime into C++ auxiliary buffers.
These behaviors are dispersed across the Python–C/C++ boundary and lose high-level semantics, making a fully automatic and sound "Python runtime values $\leftrightarrow$ C++ field offset" mapping impractical.

Therefore, \mysys adopts a pragmatic design: for a small set of core C++ types (such as \texttt{Tensor}), we manually define instantiation rules that specify how Python-level argument values (including symbolic dimensions) are embedded into the C++ memory state.
In practice, the parameter types of C++ operators are very limited—primarily \texttt{Tensor}, several container types (e.g., \texttt{vector<T>}), and scalar types—so a small number of hand-written rules suffice to support a wide range of operators across libraries.
These core abstractions are also highly stable and rarely change, making the rule set robust and low-maintenance.
Operator developers provide these rules, and the context synthesis module interprets them to construct $\mathbb{C}_{init}$ in a sound and repeatable way.

\subsection{Identify Early Return Pattern}\label{sec:design:early-return}

\begin{algorithm}[ht]
    \caption{Early-Return Edges Detection}
    \label{algo:runtime-shape-check-detection}
    \footnotesize
    \SetKwFunction{RunOnFunction}{RunOnFunction}
    \SetKwFunction{MarkEarlyReturnEdge}{MarkEarlyReturnEdge}
    \SetKwFunction{GetPhiOperand}{GetPhiOperand}
    \SetKwFunction{PushBack}{PushBack}
    \SetKwFunction{Pop}{Pop}
    \SetKwFunction{IsPhi}{IsPhi}
    \SetKwFunction{IsCall}{IsCall}
    \SetKwFunction{IsConstant}{IsConstant}
    \SetKwFunction{GetIDom}{GetIDom} 
    
    \SetKwProg{Fn}{Function}{:}{}
    
    \Fn{\RunOnFunction{$F,\;\mathcal{E},\;\mathcal{S}$}}{
        $B_{exit} \gets$ exit block of $F$\;
        \tcp{Worklist stores pairs of (Value, IncomingBlock)}
        $W \gets \GetPhiOperand(B_{exit})$\;
        \While{$W \neq \varnothing$}{
            $(v, B) \gets W.\Pop{}$\;
            \uIf{\IsPhi{$v$} defined in block $B$}{
                $W.\PushBack(\GetPhiOperand(v))$\;
            }
            \Else{
                % Found a value source v reaching exit via block B. 
                $is\_detected \gets \text{false}$\;
                $Dom \gets B$\;
                
                % Traverse up the Dominator Tree to find the shape check
                \While{$Dom \neq \text{null} \land \neg is\_detected$}{
                    $br\ x \bowtie c,\ S_{true},\ S_{false} \gets$ branch in $Dom$\;
                    
                    \If{is a conditional branch}{
                        $E_{fail} \gets \text{null}$\;
                        \If{$(\bowtie = eq \land c \in \mathcal{E}) \lor (\bowtie = neq \land c \in \mathcal{S})$}
                            {$E_{fail} \gets S_{true}$}
                        \If{$(\bowtie = eq \land c \in \mathcal{S}) \lor (\bowtie = neq \land c \in \mathcal{E})$}
                            {$E_{fail} \gets S_{false}$}
                        
                        \If{$E_{fail} \neq \text{null} \land E_{fail} \text{ dominates } B$}{
                            \If{(\IsCall{$x$} $\land\ v = x$) $\lor$ (\IsCall{$x$} $\land\ v \in \mathcal{E}$)}{
                                \MarkEarlyReturnEdge($E_{fail}$)\;
                                \RunOnFunction{callee of $x$, $\mathcal{E},\ \mathcal{S}$}\;
                                $is\_detected \gets \text{true}$\;
                            }
                        }
                    }
                    $Dom \gets \GetIDom(Dom)$\;
                }
                
                \If{$\neg is\_detected$}{
                    % Pattern 3: Tail Call (recurse)
                    \If{\IsCall{$v$}}{
                        \RunOnFunction{callee of $v$, $\mathcal{E},\ \mathcal{S}$}\;
                    }
                    % Fallback: Local check returning constant error (e.g., s % 32 != 0)
                    % Just mark the path from B to Exit as early-return
                    \ElseIf{\IsConstant{$v$} $\land\ v \in \mathcal{E}$}{
                        \MarkEarlyReturnEdge($\text{Edge}(B \to B_{exit})$)\;
                    }
                }
            }
        }
    }
\end{algorithm}

The early-return path bypasses the normal execution path, and these two paths often join at the function's ending block. 
At join points, merging values from both paths results in losing specialized constant information. 
Identifying these early-return paths allows avoiding merging value information on these paths, thereby preserving only the constant values produced on the normal execution path.
To this end, we introduce \autoref{algo:runtime-shape-check-detection} to identify and label early-return edges in the control-flow graph.
Once identified, the subsequent constant propagation pass excludes these edges, effectively pruning the error handling paths from the data-flow analysis.

At a high level, our approach relies on a reverse data-flow analysis rooted at the function's exit block. 
The fundamental principle is to trace the provenance of all values reaching the return instructions. 
In LLVM IR, a return value originates from either a compile-time constant or the result of a function call (potentially merged via \texttt{phi} nodes). 
By semantically evaluating these value sources against known error status codes, we can deduce whether a specific execution path terminates in a failure state. 
If a value source represents an error condition, the control flow edge leading to this source is identified as an early-return path.

We first define two sets of status codes, $\mathcal{E}$ and $\mathcal{S}$, representing the error and success states, respectively. 
These sets consist of compile-time constants manually configured based on the target operator library specifications (e.g., $\mathcal{S}=\{0\}$ and $\mathcal{E}=\{-1\}$).
\autoref{algo:runtime-shape-check-detection} starts by initializing a worklist with the return values from the function's exit block.
It employs the helper function \texttt{GetPhiOperand} to recursively trace the data flow: if a value corresponds to a \texttt{phi} node, its incoming operands are added to the worklist to resolve the merge logic; otherwise, the value is identified as a concrete \texttt{value source} originating from a specific basic block.

To identify local early-return edges, the algorithm analyzes the control dependencies of the identified value source by traversing the dominator tree.
It primarily looks for conditional branches that guard the execution of the return block.
If a branch compares a status variable against a constant in $\mathcal{S}$ or $\mathcal{E}$, the algorithm identifies the failure path based on the comparison logic (Patterns 1 \& 2).
However, even if no such status check is found (e.g., a local arithmetic check), if the return value source itself is a constant belonging to $\mathcal{E}$, the algorithm directly marks the path as an early return (Pattern 4).
We detail the detection process using four common patterns found in operator libraries, as illustrated in \autoref{fig:early-return-patterns}.

\begin{figure}[ht]
  \centering
  % Row 1: Pattern 1 & 2
  \begin{minipage}{0.25\textwidth}
      \centering
      % 顶格写
      \begin{lstlisting}[language=llvm, basicstyle=\scriptsize\ttfamily, frame=single, numbers=none, breaklines=true, xleftmargin=2pt, xrightmargin=2pt]
  %ret = call @check
  %c = icmp ne %ret, 0
  br %c, %exit, %nxt
exit: ; Joined block
  %v = phi i32 [%ret, %pred], ...
  ret i32 %v
      \end{lstlisting}
      \textbf{(a) Pass-through}
  \end{minipage}
  \hfill
  \begin{minipage}{0.25\textwidth}
      \centering
      % 顶格写
      \begin{lstlisting}[language=llvm, basicstyle=\scriptsize\ttfamily, frame=single, numbers=none, breaklines=true, xleftmargin=2pt, xrightmargin=2pt]
  %ret = call @check
  %c = icmp ne %ret, 0
  br %c, %exit, %nxt
exit: ; Joined block
  %v = phi i32 [-1, %pred], ...
  ret i32 %v
      \end{lstlisting}
      \textbf{(b) Explicit Error}
  \end{minipage}
  \hfill
  \begin{minipage}{0.25\textwidth}
      \centering
      % 顶格写
      \begin{lstlisting}[language=llvm, basicstyle=\scriptsize\ttfamily, frame=single, numbers=none, breaklines=true, xleftmargin=2pt, xrightmargin=2pt]
  %rem = srem i32 %s, 32
  %c = icmp ne i32 %rem, 0
  br %c, %exit, %nxt
exit: ; Joined block
  %v = phi i32 [-1, %pred], ...
  ret i32 %v
      \end{lstlisting}
      \textbf{(c) Local Constant Error}
  \end{minipage}
  \hfill
  % Row 2: Pattern 3 & 4
  \begin{minipage}{0.2\textwidth}
      \centering
      % 顶格写
      \begin{lstlisting}[language=llvm, basicstyle=\scriptsize\ttfamily, frame=single, numbers=none, breaklines=true, xleftmargin=2pt, xrightmargin=2pt]
%ret = call @check
; Direct return
; No local check
ret i32 %ret
      \end{lstlisting}
      \textbf{(d) Tail Call}
  \end{minipage}

  \Description{Simplified LLVM IR examples showing four common early return patterns: Pass-through, Explicit Error, Local Constant Error, and Tail Call.}
  \caption{Simplified LLVM IR examples of common early return patterns handled by Algorithm~\ref{algo:runtime-shape-check-detection}}
  \label{fig:early-return-patterns}
\end{figure}

\textbf{Pass-through.} 
As shown in \autoref{fig:early-return-patterns}(a), the code checks a status variable returned by a subroutine and returns it directly upon failure.
The algorithm traces the return value \texttt{\%v} back to its source, \texttt{\%ret}.
By analyzing the dominator tree, it finds the predecessor's branch condition \texttt{\%ret != 0}.
Since this implies the edge to \texttt{exit} is an error path, and the return value matches the checked variable, the incoming edge is marked as an \texttt{Early-Return}.
Subsequently, it recursively invokes \texttt{RunOnFunction} on the callee \texttt{@check}.

\textbf{Explicit Error.} 
In \autoref{fig:early-return-patterns}(b), the code checks a status variable but explicitly returns a constant error code.
Here, the value source is the constant \texttt{-1}.
The algorithm verifies that \texttt{-1} belongs to $\mathcal{E}$ and detects that the guarding branch targets the \texttt{exit} block upon failure.
Although the returned value differs from the checked variable, the semantic link via control dependency allows the algorithm to mark the edge as \texttt{Early-Return} and recursively analyze the callee.

\textbf{Local Constant Error.}
\autoref{fig:early-return-patterns}(c) illustrates a case where a local shape check (e.g., alignment check \texttt{s \% 32 != 0}) leads to an immediate error return.
The value source is the constant \texttt{-1}.
In this case, even if the algorithm does not find a status-code-based check in the dominator tree (since the check is arithmetic), it identifies that the return value \texttt{-1} belongs to the error set $\mathcal{E}$.
Consequently, the algorithm treats the path originating from this value source as a confirmed failure path and marks the edge leading to the exit block as \texttt{Early-Return}.

\textbf{Tail Call.}
\autoref{fig:early-return-patterns}(d) shows a tail call pattern with no local checks.
The return value source is \texttt{\%ret}.
Since no guarding conditional branch is found, no local edge is marked.
However, identifying that the return value comes from a call instruction allows the algorithm to recursively apply the analysis to \texttt{@check}, ensuring interprocedural soundness.


Complementing the local identification, the inter-procedural analysis is essential as status codes typically originate from nested subroutine calls.
Therefore, whenever the algorithm identifies a value source associated with a function call—whether it matches the failure branch in Patterns 1 and 2, or is a direct tail call in Pattern 3—it identifies the callee function and recursively invokes \texttt{RunOnFunction}.
This ensures that the "early return" property is correctly propagated from the innermost shape check functions up to the entry points of the operator.


\subsection{Information Propagation on Host-Side Programs and Device Kernel Specialization}\label{sec:design:propagation}

Our primary objective is to optimize the performance of device kernel functions. 
To this end, \mysys serves two distinct roles:
(1) For host-side functions, it propagates precise semantic information (pointer targets, function objects) to kernel launch points.
(2) For device-side functions, it performs kernel specialization via parameter substitution and constant folding based on the propagated context.

\subsubsection{Inter-Procedural Information Propagation for Host-Side Programs}

\mysys employs a unified inter-procedural analysis framework integrating pointer analysis, function object propagation, and scalar constant propagation.
Operating on the CFG, the analysis maintains a symbolic environment $\mathbb{E}$ and a flow-sensitive memory store $\mathbb{S}$.
We detail the handling of key program constructs below.

\paragraph{Pointer Arithmetic and Alias Analysis}
We represent a pointer value as a pair $\langle \mathcal{B}, \delta \rangle$. 
For pointer arithmetic (e.g., \texttt{getelementptr}), we update the offset $\delta$ while preserving the base set $\mathcal{B}$.
To balance precision and overhead, we bound $|\mathcal{B}| \le 5$, demoting the base to \texttt{Unknown} if this limit is exceeded.
Function pointers are similarly propagated to support indirect calls.
To refine alias analysis when bases are unknown, we leverage LLVM's Type-Based Alias Analysis (TBAA) metadata to restrict the scope of potential memory conflicts to type-compatible objects.

\paragraph{Memory Operations}
The memory store $\mathbb{S}$ is updated and accessed by three primary instructions.
Allocation instructions introduce new abstract locations into $\mathbb{S}$, managed with context-sensitive modeling.
Store instructions update the flow-sensitive store $\mathbb{S}$ local to the current basic block: writing to a pointer with a deterministic base performs a \kw{strong update} (replacement), whereas writing to an unknown base or offset triggers a \kw{weak update} (conservative invalidation/merge).
Load instructions simply retrieve the current specialization value from $\mathbb{S}$ corresponding to the accessed address.

\paragraph{Context-Sensitive Function Calls}
We employ a context-sensitive approach akin to abstract interpretation.
Each call site generates a specialized calling context based on the actual arguments and the current store $\mathbb{S}$.
The algorithm recursively analyzes the callee, using memoization to cache results for identical contexts.
Recursion is handled by tracking the call stack and returning a conservative "top" value ($\top$) upon detecting cycles, ensuring termination.

\paragraph{Control Flow Processing}
We adopt a hybrid strategy for loops, prioritizing a \kw{Per-Iteration Analysis} that simulates execution iteration by iteration.
If loop conditions become non-constant, the analysis falls back to a \kw{Maximal Fixed-Point (MFP)} iteration.
Within these modes, conditional branches determine edge liveness: if a condition evaluates to a constant, the non-taken edge is marked \texttt{Inactive}; otherwise, both branches are marked \texttt{Active}.
This liveness information is critical for pruning unreachable paths during propagation.

\paragraph{Context Merging and Early-Return Integration}
At control flow convergence points, we merge the incoming stores $\mathbb{S}$ from all \texttt{Active} predecessors and merge incoming specialization values for \texttt{PHI} instructions.
The merging applies standard lattice operations.
Crucially, this step integrates with the optimization described in \cref{sec:design:early-return}.
Any predecessor edge marked as an \texttt{early-return} path, or determined to be unreachable by the branch analysis, is treated as \texttt{Inactive}.
Consequently, the memory states and values from these inactive blocks are excluded entirely from the merge operation.
This prevents the pollution of the analysis state by imprecise or error-handling logic, significantly improving the precision of the propagated information at join points. stores $\mathbb{S}$ from all predecessors.

\subsubsection{Device Kernel Specialization}

Once the inter-procedural analysis reaches a kernel launch point, \mysys leverages the propagated context to perform kernel specialization. 
We extract concrete values from the host-side environment $\mathbb{E}$ and memory store $\mathbb{S}$ to replace the corresponding kernel parameters. 
By substituting variable accesses with constants, we resolve data dependencies and control flow decisions at compile time, transforming the generic kernel into a highly optimized, context-specific version.

For kernels invoked via driver launch APIs, the mapping from host arguments to device parameters is indirect. 
We construct the specialization context by extracting the target kernel function object from the API call. 
Using predefined ABI (Application Binary Interface) rules, we map the launch API's argument array to the kernel's formal parameters, ensuring that the constant values identified on the host side are correctly bound to the device function's context.

Following parameter substitution, we employ a tailored pipeline of standard LLVM optimization passes to exploit the newly exposed optimization opportunities. 
As implemented in our pass manager, this pipeline iteratively applies \texttt{Interprocedural Sparse Conditional Constant Propagation (IPSCCP)} and \texttt{Instruction Combining} to fold computations involving the new constants. 
Furthermore, we execute aggressive loop transformations—including \texttt{Loop Unrolling} and \texttt{Loop Invariant Code Motion (LICM)}, alongside \texttt{Global Value Numbering (GVN)} and \texttt{CFG Simplification}. 
These passes work in concert to eliminate dead code, simplify control flow, and remove redundancies, significantly reducing the runtime overhead of the specialized kernel. 

\begin{acks}
To Robert, for the bagels and explaining CMYK and color spaces.
\end{acks}

%%
%% The next two lines define the bibliography style to be used, and
%% the bibliography file.
\bibliographystyle{ACM-Reference-Format}
\bibliography{opt,pa,ref}


%%
%% If your work has an appendix, this is the place to put it.
\appendix

\section{Research Methods}

How to?

\end{document}
\endinput
%%
%% End of file `sample-manuscript.tex'.
