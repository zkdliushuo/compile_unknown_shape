
%Regular/special-issue paper – 100 to 200 words
% \IEEEtitleabstractindextext{%
\begin{abstract}

% 参数化算子 -> 提供编程接口灵活性的同时,算子内部包含了大量标量操作,用于计算并行、分块、计算访存源语等优化超参数 -> 
% 
% AI 加速芯片依赖众核

% \begin{CJK}{UTF8}{gbsn} 
% 深度学习应用中广泛调用参数化的算子,这样可以允许编程人员指定输入输出张量的形状、数据类型等信息,提供了足够的编程灵活性。
% 为了适配各种可能的参数输入,算子内部包含了大量的标量运算和分支跳转,用于根据参数取值计算并行、分块、硬件源语等优化超参数。
% 上述情况容易造成寄存器溢出、分支预测失败、指令缓存不命中等情形,使我们的目标硬件 Ascend (一种典型的被广泛应用的NPU硬件)的性能受限无法达到理论最大运算性能。
% 编译器中广泛采用的常量优化技术是避免上述情况发生的重要手段,但我们观察到,即使给定了许多参数的取值,
% 现有的编译设施在编译算子时仍做不好常量优化,导致运行时AI加速芯片上标量计算部件很容易成为性能瓶颈所在。
% 本文总结了常见的阻碍常量优化的参数化算子的程序特征,并针对性提出一种高效的算子常量优化算法。
% 在改进现有常量传播算法的同时,还引入了针对不可达语义以及内存别名的精确分析方法,最终有效缓解了算子执行时的标量性能瓶颈。
% 在当前典型的标量性能受限算子上,该方法编译的特化算子平均取得了1.1倍的性能提升,显著优于现有的最先进编译设施。
% \end{CJK}

Parameterized operators are commonly used in deep learning applications.
They allow programmers to specify tensor shapes, data types, and other properties for flexibility.
These operators involve many scalar computations and branch operations to optimize parallelism and hardware settings for various parameters. 
However, this can cause issues like register spill, branch prediction penalty, and cache misses, which limit the performance of hardware like Ascend (a commonly used NPU).
Given specific parameter values, existing compilers use constant propagation to optimize scalars and branches within operators. 
However, effective constant propagation is often disrupted by certain program branches and inaccurate data dependency analysis, preventing the optimization process.
We present an improved approach that models and addresses these challenges, leading to significant alleviation of scalar performance bottlenecks.
Our approach achieves an average 1.2x performance improvement on the typical operator whose performance is limited to scalar operations, outperforming state-of-the-art compilers.

\end{abstract}