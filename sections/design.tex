\section{System Design of \mysys}
\label{sec:design}

We propose \mysys, a highly accurate constant propagation framework that infers constant tiling parameter values from given shape parameter constants.
\mysys takes an LLVM program as input and requires a specification entry function (i.e. Host code $f$ in \autoref{fig:matmul-definition}) with its specialization shape context, $\mathbb{C}$. 
The analysis target is the memory object storing the tiling parameters, explicitly passed as a parameter to the entry function. 
%\mysys analyzes the contents of this memory object to determine constant tiling parameters.
Starting from the entry function, \mysys performs context-sensitive interprocedural analysis, propagating constants through each basic block.
At the end of the analysis, the derived constant values for the target memory object represent the inferred constant tiling parameters.

\subsection{Definitions}

The shape context $\mathbb{C}$ maps an input dimension parameter $d \in \Vec{S_D}$ to a constant $c\in \Vec{c}$.
If $d\in \text{keys}(\mathbb{C})$, then $d$ is a \kw{constant dimension}, and its dimension size is $\mathbb{C}[d]$.
The corresponding specialized values $sv \in \mathcal{SV}$ can be classified into four distinct categories: constant $c$, pointer $p$, Unknown($\bot$), or Nothing($\top$). A constant $c$ represents an instruction or memory location with a particular constant value, which can be an integer, floating point number, or function pointer.

% The former typically includes Shape APIs with unknown semantics in deep learning frameworks, as well as functions whose precise behavior is challenging to analyze.  
We assume that pointer arithmetic instructions cannot modify the base object a pointer refers to.
A base object corresponds to an allocation instruction or a global variable. 
This assumption aligns with the semantics of LLVM IR, which ensures that pointer arithmetic operations cannot access objects allocated by other instructions beyond the bounds of the base object.
When analyzing objects, $\text{Pointer(Base, offsets)}$ is often used, where
$\text{Base}$ is an allocation instruction instance or global variable, and $\text{offsets}$ is a vector. Each element is either a constant integer or $\bot$.

\begin{figure}[t]
\centering\small
%\[
\begin{tabular}{ll}
\text{Dimension size} &$d\in \Vec{S_D}$ \\
\text{Shape Context} &$\mathbb{C} := \Vec{S_D} \mapsto \Vec{c}$ \\
\text{Top-level Variable}& $v\in \mathcal{V}$ \\
\text{Memory Object}& $o\in \mathcal{O}$ \\
\text{Constant}& $c :=  \text{Integer}\ \mid \text{Float}\ \mid \text{Function Pointer} $\\
\text{Pointer}& $p := (\text{Base, Offsets}_{all}) $\\
\text{Specilized Value}& $sv := c \mid p \mid \top \mid \bot \qquad   sv\in \mathcal{SV}$ \\
\text{Environment}& $\mathbb{E} := \mathcal{V} \mapsto \mathcal{SV}$ \\
\text{Store}& $\mathbb{S} := \mathcal{O} \mapsto \mathcal{SV}$
\end{tabular}
%\]
\caption{Abstract domains of \mysys.}
\label{fig:abstract-domain}

\end{figure}

\begin{algorithm}[ht]
    \caption{Early-Return Edges Detection}
    \label{algo:runtime-shape-check-detection}
    \footnotesize
    \SetKwFunction{RunOnFunction}{RunOnFunction}
    \SetKwFunction{MarkEarlyReturnEdge}{MarkEarlyReturnEdge}
    \SetKwFunction{GetPhiOperand}{GetPhiOperand}
    \SetKwFunction{PushBack}{PushBack}
    \SetKwFunction{Pop}{Pop}
    \SetKwFunction{IsPhi}{IsPhi}
    \SetKwFunction{IsCall}{IsCall}
    \SetKwFunction{IsConstant}{IsConstant}
    \SetKwFunction{GetIDom}{GetIDom} 
    
    \SetKwProg{Fn}{Function}{:}{}
    
    \Fn{\RunOnFunction{$F,\;\mathcal{E},\;\mathcal{S}$}}{
        $B_{exit} \gets$ exit block of $F$\;
        \tcp{Worklist stores pairs of (Value, IncomingBlock)}
        $W \gets \GetPhiOperand(B_{exit})$\;
        \While{$W \neq \varnothing$}{
            $(v, B) \gets W.\Pop{}$\;
            \uIf{\IsPhi{$v$} defined in block $B$}{
                $W.\PushBack(\GetPhiOperand(v))$\;
            }
            \Else{
                % Found a value source v reaching exit via block B. 
                $is\_detected \gets \text{false}$\;
                $Dom \gets B$\;
                
                % Traverse up the Dominator Tree to find the shape check
                \While{$Dom \neq \text{null} \land \neg is\_detected$}{
                    $br\ x \bowtie c,\ S_{true},\ S_{false} \gets$ branch in $Dom$\;
                    
                    \If{is a conditional branch}{
                        $E_{fail} \gets \text{null}$\;
                        \If{$(\bowtie = eq \land c \in \mathcal{E}) \lor (\bowtie = neq \land c \in \mathcal{S})$}
                            {$E_{fail} \gets S_{true}$}
                        \If{$(\bowtie = eq \land c \in \mathcal{S}) \lor (\bowtie = neq \land c \in \mathcal{E})$}
                            {$E_{fail} \gets S_{false}$}
                        
                        \If{$E_{fail} \neq \text{null} \land E_{fail} \text{ dominates } B$}{
                            \If{(\IsCall{$x$} $\land\ v = x$) $\lor$ (\IsCall{$x$} $\land\ v \in \mathcal{E}$)}{
                                \MarkEarlyReturnEdge($E_{fail}$)\;
                                \RunOnFunction{callee of $x$, $\mathcal{E},\ \mathcal{S}$}\;
                                $is\_detected \gets \text{true}$\;
                            }
                        }
                    }
                    $Dom \gets \GetIDom(Dom)$\;
                }
                
                \If{$\neg is\_detected$}{
                    % Pattern 3: Tail Call (recurse)
                    \If{\IsCall{$v$}}{
                        \RunOnFunction{callee of $v$, $\mathcal{E},\ \mathcal{S}$}\;
                    }
                    % Fallback: Local check returning constant error (e.g., s % 32 != 0)
                    % Just mark the path from B to Exit as early-return
                    \ElseIf{\IsConstant{$v$} $\land\ v \in \mathcal{E}$}{
                        \MarkEarlyReturnEdge($\text{Edge}(B \to B_{exit})$)\;
                    }
                }
            }
        }
    }
\end{algorithm}

\subsection{Identify Early Return Pattern}


% Early return patterns hinder constant propagation by interfering with the proper merging of constant values.
The early-return path bypasses the normal execution path, and these two paths often join at the function's ending block. 
At join points, merging values from both paths results in losing specialized constant information. 
Identifying these early-return paths allows avoiding merging value information on these paths, thereby preserving only the constant values produced on the normal execution path.

As a result, we introduce Algorithm~\ref{algo:runtime-shape-check-detection} to identify and label early-return edges in the control-flow graph.
After annotating edges as either Normal or Early-Return, the following constant propagation can selectively merge incoming values only from Normal edges. All successor basic blocks reachable solely via early-return edges are excluded, thereby preventing them from affecting the join-point computations.


Algorithm~\ref{algo:runtime-shape-check-detection} detects early-return edges derived from runtime shape checks in the given function $F$. 
It starts from the exit block of $F$ and assumes the return value is a \texttt{phi} instruction node.
Each incoming pair \((v, P)\) of the \texttt{phi} corresponds to a value \(v\) from a predecessor block \(P\).
If \(v\) is a compile-time constant in the error set \(\mathcal{E}\), the predecessor block \(P\) belongs to an early-return path.
Otherwise, it belongs to a normal path.
Then, a backward propagation (\texttt{PropagateBack}) process marks corresponding labels for all edges on these paths.
Note that a normal mark will overwrite any previous early-return marks for soundness.


However, simply identifying early-return edges in a single function is not sufficient.
In some cases, the early returns may be incurred because of the invocation of a subroutine. 
Therefore, the algorithm has to recursively identify which subroutine generates the early return path.
Obviously, basic blocks that contain such sub-routines have both normal and early-return outgoing edges.
The algorithm extracts the condition value from the last conditional branch instruction in such blocks. 
Assuming the condition value is of the form \(v \bowtie s\), where \(\bowtie\ \in \{eq, neq\}\), the algorithm checks if it indicates 
\(s\) belongs to \(\mathcal{E}\) when \(\bowtie\) is \(eq\), or if \(s\) belongs to the success state set \(\mathcal{S}\) when \(\bowtie\) is \(neq\). 
If these conditions hold and \(v\) is a return value from a function call instruction, the algorithm recursively applies its analysis to that function. 
This method employs a reverse control-flow traversal to capture the complete set of reachable edges and integrates status code evaluation to accurately identify segments affected by runtime shape checks.

Algorithm~\ref{algo:runtime-shape-check-detection} relies on several helper functions.
\texttt{GetPhiOperand} extracts the operand list of a \texttt{phi} instruction in the exit block, pairing each incoming value with its predecessor block.  
\texttt{IsConstant} checks whether a value is a compile-time constant.  
\texttt{MarkNormalEdge} and \texttt{MarkEarlyReturnEdge} are used to annotate edges with their respective roles in the control-flow graph. 
\texttt{IsNormal} and \texttt{IsEarlyReturn} query the current marking of an edge.  
\texttt{ReverseReachableEdges} returns all edges that can reach a given block in reverse, allowing backward propagation of the edge property.  

\begin{algorithm}[htp]
\caption{Inter-Procedural Constant Propagation}
\label{algo:const-propagation}
\small
\SetKwFunction{MergePredecessors}{MergePredecessors}
\SetKwFunction{RunOnBasicBlock}{RunOnBasicBlock}
\SetKwFunction{MarkActiveEdge}{MarkActiveEdge}
\SetKwFunction{IsActiveEdge}{IsActiveEdge}
\SetKwFunction{ProcessSuccessors}{ProcessSuccessors}
\SetKwFunction{ExitMFPMode}{ExitMFPMode}
\SetKwFunction{MarkMFPMode}{MarkMFPMode}
\SetKwFunction{InMFPMode}{InMFPMode}
\SetKwFunction{RunOnFunction}{RunOnFunction}
\SetKwFunction{PushFront}{PushFront}
\SetKwFunction{PopFront}{PopFront}
\SetKwFunction{IsConstant}{IsConstant}
\SetKwFunction{SuccessorLiveness}{SuccessorLiveness}
\SetKwFunction{GetEdgeCondition}{GetEdgeCondition}
\SetKwFunction{IsEarlyReturnEdge}{IsEarlyReturnEdge}
\SetKwProg{Fn}{Function}{:}{}

\Fn{\RunOnFunction{$F$, $\mathbb{S}$}}{
    $\text{Worklist}\ \mathcal{W} \gets [\text{entry block of }F]$\;  % Initialize worklist
    $\mathcal{M} \gets \texttt{null}$\;
    
    \While{$\mathcal{W} \neq \emptyset$}{
        $B \gets \mathcal{W}.\PopFront{}$\;
        $\mathbb{S}_{in} \gets$ \MergePredecessors{$B, \mathbb{S}$}\;
        $\mathbb{S}_{out}, updated \gets$ \RunOnBasicBlock{$B, \mathbb{S}_{in}$}\;
        
        \If{\InMFPMode{} $\land\ B = \mathcal{M}$}{
            \ExitMFPMode{}\;
        }
        
        \If{$\neg \InMFPMode{} \lor updated$}{
            \ProcessSuccessors($B, \mathcal{W}, \mathcal{M}, \mathbb{S}_{out}$)\;
        }
        $\mathbb{S}[B] \gets \mathbb{S}_{out}$\;
    }
}

\Fn{\MergePredecessors{$B,\mathbb{S}$}}{
    $\mathbb{S}_{in} \gets \emptyset$\;
    \ForEach{$P \in \text{Predecessors}(B)$}{
        \If{\IsActiveEdge{Edge$(P\rightarrow B)$}}{
            $\mathbb{S}_{in} \gets \mathbb{S}_{in} \cap \mathbb{S}[P]$\;
        }
    }
    \Return $\mathbb{S}_{in}$\;
}
\Fn{\ProcessSuccessors{$B, \mathcal{W}, \mathcal{M}, \mathbb{S}_{out}$}}{
    \ForEach{$S \in \text{Successors}(B)$}{
        $C \gets \SuccessorLiveness(B, S, \mathbb{S}_{out})$\;  
        \If{$\neg \InMFPMode{}\land C = \texttt{unknown}$}{
            \MarkMFPMode{}\;
            $\mathcal{M} \gets \text{Direct post-dom block of } B$\;
        }
        \If{$\InMFPMode{}$}{
            \If{$S \neq \mathcal{M} \land C \neq \texttt{false}$}{
                $\mathcal{W}.\PushFront(S)$\;
                \MarkActiveEdge{Edge$(B\rightarrow S)$}\;
            }
        } \ElseIf{$C = \texttt{true}$}{
            $\mathcal{W}.\PushFront(S)$\;
            \MarkActiveEdge{Edge$(B\rightarrow S)$}\;
        }
    }
}
\Fn{\SuccessorLiveness{$B, S, \mathbb{S}_{out}$}}{
    $e \gets \text{Edge}(B \rightarrow S)$\;
    \If{\IsEarlyReturnEdge{$e$}}{
        \Return \texttt{false}\;
    }
    \tcp{Edge is a Normal path}
    $cond \gets $\GetEdgeCondition{$e$}\;
    \If{$\mathbb{S}_{out}[cond] = \texttt{true}$}{
        \Return \texttt{true}\;
    }
    \ElseIf{$\mathbb{S}_{out}[cond] = \texttt{false}$}{
        \Return \texttt{false}\;
    }
    \Else{
        \Return \texttt{unknown}\;
    }
}
\end{algorithm}

\subsection{Inter-procedural constant propagation}


In this section, we present our improved constant propagation algorithm, which integrates precise control flow and context-sensitive analysis to reduce active edges and merge points. 
As shown in Algorithm~\ref{algo:const-propagation}, it operates on the control flow graph (CFG) of a function $F$, starting from the entry block and processing blocks via a worklist $\mathcal{W}$.

For each basic block $B$, the algorithm first computes an input context $\mathbb{S}_{in}$ by merging contexts from predecessor blocks through the function \texttt{MergePredecessors}. 
The block is then processed using \texttt{RunOnBasicBlock} to yield an output context $\mathbb{S}_{out}$ and a flag $updated$ indicating whether the context has been updated.
If the context has changed or the algorithm is not in maximal fixed-point (MFP) mode, the successors of $B$ are pushed into the worklist using \texttt{ProcessSuccessors}.

The key function \texttt{RunOnBasicBlock} implements three distinct processing components:

\begin{itemize}
    \item Arithmetic Instruction Handling:
    Performs straightforward semantic-based constant propagation for fundamental operations through direct value substitution.
    \item Pointer-Centric Instruction Handling: track the pointers' access paths, as detailed in the next section. 
    \item Function Call: Update the environment $\mathbb{E}$ of the function entry basic block according to the actual parameters of the call point, and then call the function \texttt{RunOnFunction} for processing
\end{itemize}


A key component of our approach is the \texttt{SuccessorLiveness} function, which evaluates the reachability of each edge from $B$ to a successor $S$ based on the specialized context $\mathbb{S}_{out}$ to determine the liveness of successor block $S$ from current block $B$.
This function returns a value $C \in \{\texttt{true}, \texttt{false}, \texttt{unknown}\}$, reflecting whether the branch is definitely reachable, definitely unreachable, or uncertain. 
In non-MFP mode, only successors with $C=\texttt{true}$ are processed.
If $C$ is \texttt{unknown}, the algorithm switches to MFP mode and sets the final merge point $\mathcal{M}$ to the direct post-dominating block of $B$.
In MFP mode, every successor $S\neq\mathcal{M}$ with $C\neq\texttt{false}$ is added to the worklist, and the corresponding edge is marked as active.

This design offers finer control over path merging by filtering out provably unreachable edges. 
The precision of constant propagation is improved by avoiding the propagation of imprecise contexts from definitely unreachable paths.
Moreover, the adaptive switch to MFP mode ensures that the analysis remains conservative when the branching target is uncertain.
Enhanced memory alias analysis further increases the accuracy of condition evaluation, leading to better overall performance in our constant propagation.

\subsection{On-the-fly Point-to Propagation Algorithm}

Accurate constant propagation recovers constant values for many index operands in pointer arithmetic instructions(e.g. \texttt{Field} instructions).
A pointer targeting a specific field can be represented by an access path, a sequence of constant indices.
This access path constrains the set of potential target objects.
When the base object is uncertain, these constant indices act as constraints to reduce the number of candidate base objects.
To track the pointers' access paths, we first formalize the data flows in LLVM IR instructions into a set of access path propagation rules.
The function \texttt{RunOnBasicBlock} in \autoref{algo:const-propagation} adopts these rules for each instruction and maintains this information in the environment $\mathbb{E}$ and store $\mathbb{S}_{out}$.
To implement these rules, we maintain the structure of each allocated memory object and define some auxiliary functions for the propagation.

\myparagraph{Auxiliary Functions for Access Path Propagation.} 
We use function \(\mathcal{F}(p)\) to query the set $tar(p)$ of all the may-point-to field objects of the pointer $p$.
The function \(\mathcal{F}\) calculates the set based on a pair \((o,\text{Offsets})\) (obtained from \(\mathbb{E}[p] \mapsto (o,\text{Offsets})\)). 
\(\mathcal{F}\) is implemented in three different ways.
First, if both \(o\) and \(\text{Offsets}\) are fully determined (i.e., all values are constant), then \(\mathcal{F}(o,\text{Offsets})\) returns a singleton set containing the unique field object pointed to by the complete access path.
Second, if \(o\) is determined but some indices in \(\text{Offsets}\) are uncertain (i.e., include \(\bot\)), then \(\mathcal{F}(o,\text{Offsets})\) traverses the hierarchical structure of \(o\) to locate all field objects that correspond to the known constant indices in the access path.
Lastly, if \(o\) is also uncertain, then \(\mathcal{F}(o,\text{Offsets})\) uses the constant indices present in \(\text{Offsets}\) as constraints and searches over all allocated base objects whose hierarchical layout can accommodate an access path matching the known indices.
In this case, the union of the corresponding field objects from all matching base objects is returned.

Besides, the function \(\text{OffsetsOf}\) takes a field object $o$ and computes the complete access path (a sequence of constant indices) from the root of the base object to $o$.
For example, the access path to \(\texttt{o1.p[1].x}\) at line 12 in \autoref{fig:early-return-code} would be \(o1, [0, 1, 0]\), where each index corresponds to a field or array access.
The byte offset operation $o+i$ calculates a new pointer. 
It computes a target address by adding byte offset $i$ to the address of $o$. 
Then, to resolve ambiguity where multiple typed objects (like a struct and its first field) may share this address, the operation normalizes the result by returning the most fine-grained field object residing at the target address. 
This corresponds to the leftmost leaf in the type hierarchy rooted at $o$.

\begin{figure}
    \centering
    \includegraphics[width=\linewidth]{figures/motivation/aggregate-for-tilingdata.png}
    \caption{Example of how to compute the point-to set and the access path for the pointer \texttt{ptr}. Symbols \texttt{PartData} and \texttt{algo} are defined in \autoref{fig:early-return-code}}
    \label{fig:example-point-to-analysis}
\end{figure}

The intersection operator \(\cap\) is used to calculate the overlap between two access paths. 
It returns the shared base object and constant indices, and introduces uncertainty (\(\bot\)) if any index is unknown or mismatched.
When intersecting two access paths of different lengths and different bases, the result is \(\bot\), denoting unknown access path with unknown side-effects.


\begin{figure}[htbp]
\centering
\footnotesize
\[
\resizebox{\linewidth}{!}{$
\begin{array}{l}
[\text{AddrOf}]
\dfrac{p \overset{AddrOf}{\longleftarrow} o}
      {\mathbb{E}[p] \mapsto \bigl(o,\langle\rangle\bigr)}
\quad
[\text{Copy}]
\dfrac{p \overset{Copy}{\longleftarrow} q,\ \mathbb{E}[q] \mapsto \bigl(\text{Base},\text{Offsets}_q\bigr)}
      {\mathbb{E}[p] \mapsto \bigl(\text{Base},\text{Offsets}_q\bigr)}
\\[1mm]
[\text{Field}]
\dfrac{p \overset{Field_i}{\longleftarrow} q,\ \mathbb{E}[q] \mapsto \bigl(o,\text{Offsets}_q\bigr),\ |\mathcal{F}(q)|=1}
      {\mathbb{E}[p] \mapsto \bigl(o,\text{OffsetsOf}(\mathcal{F}(q)+i)\bigr)}
\\[1mm]
[\text{Field}]
\dfrac{p \overset{Field_i}{\longleftarrow} q,\ \mathbb{E}[q] \mapsto \bigl(\text{Base},\text{Offsets}_q\bigr),\ |\mathcal{F}(q)|>1}
      {\mathbb{E}[p] \mapsto \Bigl(\text{Base},\bigcap_{o'\in \mathcal{F}(q)} \bigcap_{i'\in \mathbb{E}(i)} \text{OffsetsOf}(o'+i')\Bigr)}
\\[1mm]
[\text{Phi}]
\dfrac{r \leftarrow \phi(p,q),\ p = \bigl(o_p,\text{Offsets}_p\bigr),\ q = \bigl(o_q,\text{Offsets}_q\bigr),\ o_p = o_q}
      {\mathbb{E}[r] \mapsto
              \bigl(o_p,\text{Offsets}_p \cap \text{Offsets}_q\bigr) }
\\[1mm]
[\text{Phi}]
\dfrac{r \leftarrow \phi(p,q),\ p = \bigl(o_p,\text{Offsets}_p\bigr),\ q = \bigl(o_q,\text{Offsets}_q\bigr),\ o_p \neq o_q}
      {\mathbb{E}[r] \mapsto 
              \bigl(\bot,\text{Offsets}_p \cap \text{Offsets}_q\bigr)}
\\[1mm]
[\text{Load}]
\dfrac{p \overset{Load}{\longleftarrow} q,\ \mathbb{E}[q] \mapsto \bigl(o,\text{Offsets}_q\bigr),\ |\mathcal{F}(q)|=1}
      {\mathbb{E}[p] \mapsto \mathbb{S}[\mathcal{F}(q)]}
\\[1mm]
[\text{Load}]
\dfrac{p \overset{Load}{\longleftarrow} q,\ \mathbb{E}[q] \mapsto (o, \text{Offsets}_q),\ |\mathcal{F}(q)|>1}
      {\mathbb{E}[p] \mapsto \bigcap_{o' \in \mathcal{F}(q)} \mathbb{S}[o']}
\\[1mm]
[\text{Store}]
\dfrac{p\overset{Store}{\longleftarrow}q,\ \mathbb{E}[p] \mapsto \bigl(o,\text{Offsets}_p\bigr),\ |\mathcal{F}(q)|=1}
      {\mathbb{S}[\mathcal{F}(p)] \gets \mathbb{E}[q]}
\\[1mm]
[\text{Store}]
\dfrac{p\overset{Store}{\longleftarrow}q,\ \mathbb{E}[p] \mapsto \bigl(o,\text{Offsets}_p\bigr),\ |\mathcal{F}(q)|>1}
      {\forall\, o' \in \mathcal{F}(p):\ \mathbb{S}[o'] \gets \bot}
\end{array}
$}
\]
\caption{Accurate access path propagation rules trace the data flow in LLVM IR instructions based on constant offset indexes.}
\label{eq:access-path-propagation-rules}

\end{figure}


\myparagraph{Access Path Propagation Rule Statement.}

\autoref{eq:access-path-propagation-rules} lists access path propagation rules for five kinds of instructions with respect to pointer variables.
Our analysis realizes these rules in a field-sensitive, flow-sensitive, yet context-insensitive manner to precisely trace access paths throughout the program. To ensure accuracy in loops, the analysis iteratively applies these rules until a fixed point is reached, thereby capturing stable propagation results across all control flow paths.

\begin{itemize}
\item AddrOf: When a pointer variable is assigned the address of an object, its access path is initialized with the object and an empty offset sequence. This represents a direct reference to the base object.

\item Copy: A copy operation propagates the exact access path from the source pointer to the target. This ensures aliasing relationships are preserved accurately.

\item Field Access:  
If the pointer refers to a unique object ($|\mathcal{F}(q)| = 1$), the field index is appended to the offset path (\textit{strong update}).  
If multiple targets exist ($|\mathcal{F}(q)| > 1$), we intersect their access paths after extension (\textit{weak update}).

\item Phi:  
If both operands share the same base object ($o_p = o_q$), intersect their offset sets (\textit{strong merge}).  
If not ($o_p \ne o_q$), set the base to $\bot$ and intersect the offsets (\textit{weak merge}).

\item Load:  
For a unique target ($|\mathcal{F}(q)| = 1$), load the stored access path directly (\textit{strong read}).  
Otherwise, intersect the access paths of all possible targets (\textit{weak read}).

\item Store:  
If the target is unique ($|\mathcal{F}(p)| = 1$), update the field with the source’s access path (\textit{strong update}).  
For multiple targets, assign $\bot$ to all possible destinations (\textit{weak update}).
\end{itemize}

% For example, assume calculating a pointer \texttt{ptr} pointing to one member \texttt{o2.t.p[i]} inside the \texttt{o2}~(i.e. \texttt{algo}) in \autoref{fig:early-return-code}, where \texttt{i} is a variable.
% The LLVM type structure of \texttt{o2} and the corresponding LLVM IR instructions are shown in \autoref{fig:example-point-to-analysis}.
% Although \texttt{i} uncertain 


Consider the example in \autoref{fig:example-point-to-analysis}, which shows the calculation of a pointer \texttt{ptr} to the field \texttt{algo.t.p[i]}, where \texttt{i} is a dynamic variable.
Fortunately, under the assumption that the input programs are well-formed and do not exhibit undefined behavior from out-of-bounds memory access, we can constrain the possible values of the index \texttt{i} to range \(\{0,1,2\}\) through the type information of \texttt{\%tmp}.
The range infers that \texttt{\%ptr} may points to \texttt{0.x}, \texttt{1.x}, and \texttt{2.x}.
By intersecting the access paths to these possible pointees, we get a single, summarized abstract path: $\mathbb{E}[\text{\%ptr}] = (\text{o2}, [0, 0, 0, 0, \bot, 0])$.
This precision effectively constrains the side effects of any write through \texttt{ptr}, preventing it from affecting unrelated fields and preserving their constant values.
