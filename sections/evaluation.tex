\section{Evaluation}
\label{sec:eval}

Our evaluation mainly focuses on the following questions:
\begin{itemize}
    \item Can \mysys enhance performance across a diverse set of operators and neural networks on Ascend devices?
    \item Can \mysys effectively infer more invariable parameters for kernels with a reasonable compilation overhead compared to current compilers?
    \item Can \mysys be useful for accelerators beyond Ascend which almost not have scalar bottlenecks, such as GPUs?
\end{itemize}
Apart from the above main questions, we evaluate the detailed performance metrics differnce brought by \mysys, the performance improvement under different shape configurations.

\subsection{Experimental Setup}
\label{sec:eval:setup}

\paragraph{Hardware and Software Platforms}
The main evaluation is on a platform equipped with an Ascend 910B1 NPU and an Intel Core i7-8700 CPU.
The version of Ascend Compute Architecture for Neural Networks(CANN) is 8.0.RC1.
The version of LLVM is 15.0.5 and the version of PyTorch is 2.6.0.
To ensure accuracy, the reported data represents the average of the last 90 runs out of 100 total executions.

\paragraph{Benchmarks}
\label{sec:eval:bench}
To ensure a comprehensive evaluation, we construct a benchmark suite consisting of representative operators derived from mainstream deep learning models, including Large Language Models (e.g., Llama, Qwen), Computer Vision models (e.g., ResNet, ViT), and Mixture-of-Experts (MoE) architectures.
These operators cover various computation patterns and dynamic shape scenarios.

We place particular emphasis on \textbf{fused operators in LLM serving scenarios}, such as \texttt{FlashAttention}, \texttt{FeedForward} (FFN), and \texttt{BatchMatMul}.
In modern LLM serving systems (e.g., vLLM, TGI), techniques like continuous batching introduce highly dynamic batch sizes during the decoding phase, while variable sequence lengths are prevalent in the prefill phase.
Optimizing these kernels is critical as they dominate the end-to-end inference latency.
Conversely, we exclude simple element-wise (pointwise) operators from our main evaluation.
Such operators typically involve minimal scalar arguments (often just tensor metadata) and are memory-bound, leaving little room for the scalar-related optimizations that \mysys targets.
\autoref{tab:dynamic_operators} summarizes the categorization and configurations of the evaluated operators.

\begin{table*}[t]
\centering
\caption{Overview of dynamic shape operator configurations used in experiments.}
\label{tab:dynamic_operators}
% 使用 resizebox 确保表格适应文本宽度
\resizebox{\textwidth}{!}{%
\begin{tabular}{lll}
\toprule
\textbf{Operator Category} & \textbf{Representative Operators} & \textbf{Layout \& Dynamic Dimension (Range)} \\
\midrule
\multirow{3}{*}{Normalization} & \texttt{LayerNorm}, \texttt{RmsNorm}, \texttt{AddRmsNorm} & Layout: \texttt{[B, S, D]}, Dynamic: \textbf{S} (Sequence) in \texttt{[128, 1024]} or \textbf{B} (Batch) in \texttt{[1, 8]} \\
& \texttt{BatchNormV3}, \texttt{GroupNormSilu} & Layout: \texttt{[N, C, H, W]}, Dynamic: \textbf{N} (Batch) in \texttt{[1, 8]} \\
& \texttt{GroupNormGrad} & Layout: \texttt{[N, C, H, W]}, Dynamic: \textbf{C} (Channel) in \texttt{[48, 80]} \\
\midrule
\multirow{2}{*}{Attention \& RNN} & \texttt{MultiHeadAttention}, \texttt{DynamicRNN} & Layout: \texttt{[B, S, D]}, Dynamic: \textbf{S} (Sequence) in \texttt{[128, 1024]} \\
& \texttt{RotaryPositionEmbedding} & Layout: \texttt{[B, H, D, S]}, Dynamic: \textbf{S} (Sequence) in \texttt{[64, 256]} \\
\midrule
\multirow{2}{*}{Matrix Multiplication} & \texttt{MatMulV3}, \texttt{MatmulAllReduce} & Layout: \texttt{[M, K]}, Dynamic: \textbf{M} in \texttt{[64, 8192]} \\
& \texttt{BatchMatMulV3}, \texttt{GroupedMatmul} & Layout: \texttt{[B, M, K]}, Dynamic: \textbf{B} (Batch) in \texttt{[1, 16]} \\
\midrule
Pooling \& Resampling & \texttt{AvgPool3D}, \texttt{Upsample} (e.g., \texttt{Bicubic}), \texttt{GridSample} & Layout: \texttt{[N, C, H, W]} or \texttt{[N, C, D, H, W]}, Dynamic: \textbf{N} (Batch) in \texttt{[1, 8]} \\
\midrule
\multirow{2}{*}{MoE Layers} & \texttt{MoeInitRouting}, \texttt{MoeGating} & Layout: \texttt{[B, S, D]}, Dynamic: \textbf{B} (Batch) in \texttt{[1, 8]} \\
& \texttt{MoeTokenPermute}, \texttt{Moe} (Grad ops) & Layout: \texttt{[T, D]} (T=Tokens), Dynamic: \textbf{T} in \texttt{[80, 960]} \\
\midrule
\multirow{2}{*}{Element-wise \& Data} & \texttt{Foreach} (e.g., \texttt{Add}, \texttt{Log}), \texttt{Gather} & Layout: \texttt{[N, ..., ]}, Dynamic: \textbf{N} (Batch/Dim) in \texttt{[1, 9]} \\
& \texttt{IsInf}, \texttt{ScatterElements} & Layout: \texttt{[..., S, ...]}, Dynamic: \textbf{S} (Sequence) in \texttt{[128, 1024]} \\
\bottomrule
\end{tabular}
}
\end{table*}


\subsection{Performance Results}



\begin{figure}
    \centering
    \includegraphics[width=0.9\linewidth]{figures/evaluation/model-speedup/multi_model_speedup.png}
    \caption{Overall normalized performance of \mysys compared to PyTorch baseline based on CANN compiler across various dynamic shape operators.}
    \Description{Overall normalized performance of \mysys compared to PyTorch baseline based on CANN compiler across various dynamic shape operators.}
    \label{fig:eval:model-overall}
\end{figure}