\section{Evaluation}
\label{sec:eval}

Our objective is to demonstrate that \mysys effectively optimizes dynamic-shape tensor operators on accelerators, outperforming the state-of-the-art.
We address the following research questions:

\begin{itemize}
    \item \textbf{RQ1} Can \mysys enhance the performance of widely-used dynamic shape operators on accelerators?
    \item \textbf{RQ2} Can \mysys effectively infer invariable parameters for kernels with a reasonable compilation overhead?
\end{itemize}

% RQ1: \subsection{Performance Results} subsec.B
% RQ1: \subsection{Performance Analysis} subsec.C

\subsection{Experimental Setup}
\label{sec:eval:setup}

\paragraph{Hardware and Software Platforms}
\mysys's evaluation covers two hardware platforms running Linux-based operating systems: an Nvidia RTX 4080 GPU and an Ascend 910B1 NPU(\autoref{tab:hardware}). 
For the GPU platform, we utilize CUDA toolkit (v12.6) and GPU library FlashInfer (v0.2.5).
For the NPU platform, we employ CANN SDK (v8.0.RC1) and an operator library~\cite{ascend_cannopsadv} customized for LLMs.
To ensure accuracy, the reported data represents the average of the last 90 runs out of 100 total executions.


\begin{table}[htbp]
\centering
\footnotesize
\caption{Specifications of experimental hardware platforms}
\label{tab:hardware}
\resizebox{\linewidth}{!}{%
\begin{tabular}{@{}ccc@{}}
\toprule
\textbf{Feature} & \textbf{GPU Server} & \textbf{NPU Server} \\ \midrule
Operating System & Ubuntu 24.04 LTS & Ubuntu 20.04 LTS \\
\hdashline 
CPU & Intel Xeon Gold 6348 & Intel Core i7-8700 \\
Host Memory & 512 GB & 32 GB \\
\hdashline 
Accelerator Card & NVIDIA RTX 4080 & Huawei Ascend 910B1 \\
Device Memory & 16 GB & 64 GB \\
Computing Core & 76 SMs & 24 DaVinci Cores \\ 
Processing Units & Tensor, CUDA Core & Cube, Vector, Scalar Units \\ \bottomrule
\end{tabular}%
}
\end{table}


% \begin{table}[ht]
% \centering
% \caption{Key Dimensions and Operator Specifications}
% \label{tab:dimensions-and-operators}
% \resizebox{\linewidth}{!}{%
% \begin{tabular}{ccc}
% \toprule
% \multicolumn{3}{c}{\textbf{Notations of Key Dimensions}} \\
% \midrule
% \textbf{Symbols} & \textbf{Description} & \textbf{Value} \\
% \midrule
% $B$ & Batch size & 4 \\
% $N$ & Number of heads & 128 \\
% $S$ & Sequence length & $\{32,64,...,2048\}$ (step=32) \\
% $D$ & Dimension per head & 64 \\
% $|S|$ & Number of possible shapes & 10 \\
% \midrule
% \multicolumn{3}{c}{\textbf{Operator Specifications}} \\
% \midrule
% \textbf{Abbr.} & \textbf{Operator Name} & \textbf{Input Shapes} \\
% \midrule
% DFA & Decoding FlashAttention & [B, N, 1, D], [S, N, D], [S, N, D] \\
% PFA & Prefill FlashAttention & [B, S, 8192], [B, S, 8192], [B, S, 8192] \\
% \bottomrule
% \end{tabular}%
% }
% \end{table}

\renewcommand{\arraystretch}{1.1}
% \begin{table}[htbp]
% \begin{threeparttable}
% \centering
% \caption{Taxonomy and Configurations of Key Shape Dimensions for LLM Attention Kernels.}
% \label{tab:attention-dimensions-definition}
% \footnotesize
% \begin{tabularx}{\linewidth}{
%     l
%     >{\raggedright\arraybackslash}p{0.85cm} % Column for Symbols
%     >{\raggedright\arraybackslash}X % Column for Description
%     l
% }
% \toprule
% \textbf{Type} & \textbf{Notation} & \textbf{Notation Description} & \textbf{Value} \\
% \midrule
% \multirow{5}{*}{\parbox{0.85cm}{\centering Invariable}}
% & $N_{qo}$& \textbf{Number of query heads} & 16/32/64/128 \\
% \cmidrule(lr){2-4}
% & $N_{kv}$& \textbf{Number of key/value heads} & 2/4/8/16/32 \\
% \cmidrule(lr){2-4}
% & $D$ & \textbf{Encoding length per head}. & 32/96/128/192 \\
% % \cmidrule(lr){2-4}
% % & $D_{pos}$ & \textbf{Positional encoding length}. & 32/96/128/192 \\
% \cmidrule(lr){2-4}
% & $B$ & \textbf{Batch size}. & 1/4/16/32 \\
% \midrule
% \multirow{4}{*}{\parbox{0.85cm}{\centering Variable}}
% & $S_{qo}$\tnote{*} & \textbf{Query Sequence Length} &\\
% & & \textit{prefill:} length of the prompt & [32, 2048, 32] \\
% & & \textit{decoding:} equals to 1 & 1 \\
% \cmidrule(lr){2-4}
% & $S_{kv}$\tnote{*} & \textbf{Key/Value Sequence Length} &\\
% & & \textit{prefill:} length of the prompt & [32, 2048, 32] \\
% & & \textit{decoding:} length of tokens so far & [32, 2048, 32] \\
% \bottomrule
% \end{tabularx}
% \begin{tablenotes}
% \item[*] \small \textit{Note:} In batched inference, operators require an array of the exact logical length of each sequence. For simplicity, we use the symbols $S_{qo}$ and $S_{kv}$ to denote the logical length of a representative sequence. The physical memory for the KV cache is allocated in discrete blocks (pages) of a fixed size (e.g., 16 tokens). Therefore, the allocated memory for a sequence is a multiple of this page size, while the kernel computation relies on the precise logical sequence length $S_{kv}$.
% \end{tablenotes}

% \end{threeparttable} % 结束 threeparttable 环境
% \end{table}

\begin{table}[htbp]
\begin{threeparttable}
\centering
\caption{Key shape dimensions and their value domain.}
\label{tab:attention-dimensions-definition}
\footnotesize
\begin{tabularx}{\linewidth}{
    >{\raggedright\arraybackslash}p{0.8cm} % Category column
    >{\raggedright\arraybackslash}p{0.8cm} % Symbol column
    >{\raggedright\arraybackslash}X      % Description column
    >{\raggedright\arraybackslash}p{2.3cm} % Value column
}
\toprule
\textbf{Category} & \textbf{Notation} & \textbf{Description} & \textbf{Values} \\
\midrule
\multirow{6}{*}{\parbox{0.85cm}{\centering Invariable}} & $B$ & \textbf{Batch size} & 1/4/16/32 \\
\cmidrule(lr){2-4}
& $N_{q}$ & \textbf{Query heads number}. & 16/32/64/128 \\
\cmidrule(lr){2-4}
& $N_{kv}$\tnote{*} & \textbf{Key/value heads number} & 1/4/8/16/32 \\
\cmidrule(lr){2-4}
& $D$ & \textbf{Dimension size per head} & 64/128/192/256 \\
\cmidrule(lr){2-4}
& $D_{pe}$ & \textbf{Position encoding dimension per head} & 32/64/128 \\
\midrule
\multirow{4}{*}{\parbox{0.85cm}{\centering Variable}} & $S_{qo}$\tnote{**} & \textbf{Query Sequence Length} & \textit{Prefill:} [32, 4096] \newline \textit{Decode:} 1 \\
\cmidrule(lr){2-4}
& $S_{kv}$\tnote{**} & \textbf{Key/Value Sequence Length} (total tokens) & \textit{Prefill:} [32, 4096] \newline \textit{Decode:} [32, 4096] \\
\bottomrule
\end{tabularx}

\begin{tablenotes}
\item[*] \small The relationship between $N_q$ and $N_{kv}$ defines the attention type: \textbf{Multi-Head (MHA)} if $N_{kv} = N_q$; \textbf{Grouped-Query (GQA)} if $1 < N_{kv} < N_q$; \textbf{Multi-Query (MQA)} if $N_{kv} = 1$.
\item[**] \small In batched inference, operators require the exact logical length of each sequence. For simplicity in this context, we use the logical lengths $S_{qo}$ and $S_{kv}$ to represent the variable dimensions.
\end{tablenotes}
\end{threeparttable}
\end{table}

% \begin{table}[htbp]
% \centering
% \caption{Value of Key Shape Dimensions for some representative LLMs.}
% \label{tab:dimensions-value-domain}
% \footnotesize
% % 确保导言区有 \usepackage{pifont} 用于 \ding 命令
% \begin{tabularx}{\linewidth}{l X c c c c}
% \toprule
% \textbf{ID} & \textbf{Model} & \textbf{Attention Type} & $\boldsymbol{N_{qo}}$ & $\boldsymbol{N_{kv}}$ & $\boldsymbol{D}$ \\
% \midrule
% \ding{172} & LLaMA-2 7B & MHA & 32 & 32 & 128 \\
% \ding{173} & LLaMA-3 8B & GQA & 32 & 8 & 128 \\
% \ding{174} & Qwen2 72B& GQA & 64 & 8 & 128 \\
% \ding{175} & Google Gemma 7B & MHA & 16 & 16 & 192 \\
% \ding{176} & Microsoft Phi-3 3.8B & GQA & 32 & 4 & 96 \\
% \ding{175} & DeepSeek-V2 236B & MLA & 128 & 1 & 32 \\
% \bottomrule
% \end{tabularx}
% \end{table}


\begin{table}[htbp]
\centering
\begin{threeparttable}
    
\caption{Benchmark settings of key shape dimensions}
\label{tab:benchmark-setting-attention-types}
\footnotesize
\begin{tabularx}{\linewidth}{
  l                % Left-aligned column for model name
  >{\centering\arraybackslash}X                % Left-aligned, expandable column for attention type
  c                % Centered column for N_q
  c                % Centered column for N_kv
  c                % Centered column for D_head
  c                % Centered column for D_rope
}
\toprule
\textbf{ID} & \textbf{Attention Type} & \textbf{$N_{q}$} & \textbf{$N_{kv}$} & \textbf{$D$} & \textbf{$D_{pe}$} \\
\midrule
\ding{172} & MHA & 32 & 32 & 128 & 0  \\ % LLaMA-2 7B \\
\ding{173} & MHA & 16 & 16 & 256 & 0  \\ % Google Gemma 7B \\
\ding{174} & GQA & 32 & 8 & 128 & 0 \\ % LLaMA-3 8B \\
\ding{175} & GQA & 32 & 4 & 128 & 0 \\ % Qwen2 72B \\
% \ding{176} & GQA & 32 & 4 & 96 & 0 \\ % Microsoft Phi-3 3.8B \\
\ding{176} & MQA & 32 & 1 & 512 & 64 \\ % MLA (Conceptual) \\
\bottomrule
\end{tabularx}


\end{threeparttable}
\end{table}
% \multicolumn{2}{p{0.95\linewidth}}{\raggedright\small%
% \textit{\textbf{Implementation Note on Paged KV Cache:} Underlying operator implementations often use a Paged KV Cache mechanism. Instead of a single dense tensor of length $S_{kv}$, the operator takes an array of pointers or indices indicating the specific physical memory pages for each sequence in a batch. For simplicity in this context, we use the logical lengths $S_{qo}$ and $S_{kv}$ to represent the variable dimensions.}
% }

\paragraph{Benchmarks} 
Practical LLM inference typically consists of two distinct computational phases: prefill and decoding. 
We evaluate two representative operators extracted from the two phases, Prefill FlashAttention (PFA, attention optimized for prefill) and Decoding FlashAttention (DFA, attention optimized for decoding).
The \autoref{tab:attention-dimensions-definition} table specifies invariable and variable dimension parameters and their value domain.
\autoref{tab:benchmark-setting-attention-types} lists five attention types with different settings, as the research targets in this section.

We compare \mysys with the following three kinds of compilation approaches:

\begin{itemize}
    \item \textbf{Dynamic Compilation (Dynamic)}: Operator implementation supporting generic shape computation without specialization
    \item \textbf{Manual Dispatch(Manual)}: Manually enumerate several key kernel parameters to specialize kernels
    \item \textbf{Just-In-Time Compilation (JIT)}: Generates optimized kernels for each possible input shape at runtime
\end{itemize}

Given that the decoding phase notably accounts for over 80\%~\cite{zhong2024distserve,agrawal2024taming} of total inference latency in typical deployment scenarios, we particularly emphasize optimization effectiveness for decoding-stage operators, i.e., the DFA operator.


% 我们在不同大小的 S (16~160) 上对DFA,PFA,MM 三个算子进行了多组实验,实验结果如图八所示。其中 DFA 算子通过 TilingInfer 编译优化达到了 20% 左右的加速,这主要来自于类似图3中展示的初始化优化。在初始化阶段,由于 DFA 算子具有许多可配置的参数,主要是对各种参数的赋值和少量的计算。计算所涉及到的指令延迟大多为 1 周期,但赋值涉及到的 load / store 延迟要远远大于计算指令。如果发生了 cache miss ,load / store 的延迟可以达到数百周期。经过了 Tiling Infer 的优化后,部分参数可以被优化为常量,相应的 load / store 可以被消除,从而达到了可观的加速。
\subsection{Kernel Optimization Results (\textbf{RQ1})}
% \begin{table}[htbp]
\centering
\caption{Performance Comparison: Speedup and Metrics across \textbf{Baseline} (B), \textbf{LLVM-spec} (L), and \textbf{\mysys} (T) methods for benchmarked operators. 
Best values are bolded (higher is better for speedup; lower is better for other metrics).}
\label{tab:performance_metrics}
\resizebox{\textwidth}{!}{%
\begin{tabular}{lcccccccccccccccccc}
\toprule
\multirow{2}{*}{\textbf{Op}} & \multicolumn{3}{c}{\textbf{Speedup} ($\uparrow$)} & \multicolumn{3}{c}{\textbf{I Cache Miss (\%)} ($\downarrow$)} & \multicolumn{3}{c}{\textbf{Scalar Ratio (\%)} ($\downarrow$)} & \multicolumn{3}{c}{\textbf{MTE2 Ratio (\%)} ($\downarrow$)} & \multicolumn{3}{c}{\textbf{ Ratio (\%)} ($\downarrow$)} & \multicolumn{3}{c}{\textbf{Binary Size(KB)} ($\downarrow$)} \\
\cmidrule(lr){2-4} \cmidrule(lr){5-7} \cmidrule(lr){8-10} \cmidrule(lr){11-13} \cmidrule(lr){14-16} \cmidrule(lr){17-19}
 & \textbf{B} & \textbf{L} & \textbf{T} & \textbf{B} & \textbf{L} & \textbf{T} & \textbf{B} & \textbf{L} & \textbf{T} & \textbf{B} & \textbf{L} & \textbf{T} & \textbf{B} & \textbf{L} & \textbf{T} & \textbf{B} & \textbf{L} & \textbf{T} \\
\midrule
Pool3D & 1.00 & 1.00 & 1.00 & 0.00 & 0.00 & 0.00 & 97.27 & 97.21 & \textbf{97.14} & 45.01 & 45.01 & 45.01 & 0.00 & 0.00 & 0.00 & 13.35 & 13.35 & \textbf{13.24} \\
BMM & 1.00 & \textbf{1.02} & \textbf{1.02} & \textbf{3.12} & 3.42 & 3.41 & 84.70 & 84.74 & \textbf{84.68} & \textbf{42.66} & 47.10 & 45.95 & 0.00 & 0.00 & 0.00 & 101.38 & \textbf{100.84} & \textbf{100.84} \\
BN & 1.00 & \textbf{1.05} & \textbf{1.05} & 0.00 & 0.00 & 0.00 & \textbf{96.60} & 96.69 & 96.82 & 64.86 & 66.16 & \textbf{60.02} & 22.94 & \textbf{22.13} & 22.44 & 15.73 & \textbf{15.61} & \textbf{15.61} \\
FFN & 1.00 & 1.01 & \textbf{1.08} & 3.89 & 4.12 & \textbf{0.00} & 84.72 & 84.52 & \textbf{83.60} & 6.42 & 7.24 & \textbf{5.49} & \textbf{0.10} & 0.11 & 0.11 & 68.79 & 68.89 & \textbf{54.91} \\
GN & 1.00 & 1.16 & \textbf{1.17} & 0.80 & 1.20 & \textbf{0.00} & 99.10 & 98.94 & \textbf{98.90} & 24.98 & \textbf{18.46} & 18.83 & 14.20 & \textbf{9.29} & 9.31 & 220.76 & 220.56 & \textbf{193.80} \\
D-Attn & 1.00 & 0.99 & \textbf{1.20} & 4.45 & 5.51 & \textbf{0.00} & 77.98 & 77.08 & \textbf{74.43} & 6.42 & 6.44 & \textbf{4.91} & 2.46 & \textbf{1.60} & 1.79 & 58.30 & 58.59 & \textbf{45.05} \\
RMS & 1.00 & 0.98 & \textbf{1.08} & \textbf{3.00} & 3.01 & 3.36 & 97.89 & 97.82 & \textbf{88.57} & 6.83 & 7.27 & \textbf{6.72} & \textbf{2.90} & 3.02 & 3.13 & 36.32 & 36.32 & \textbf{33.23} \\
P-MM & 1.00 & 1.00 & 1.01 & 1.36 & 1.40 & \textbf{1.35} & 72.02 & 72.04 & \textbf{71.61} & 47.58 & \textbf{47.23} & 48.50 & 0.00 & 0.00 & 0.00 & 82.55 & 82.55 & 82.55 \\
M-SM & 1.00 & 0.96 & \textbf{1.01} & 24.38 & 23.42 & \textbf{22.61} & \textbf{97.39} & 98.06 & 97.93 & 5.44 & 4.36 & \textbf{1.41} & \textbf{0.90} & 1.29 & 1.12 & 79.31 & 77.79 & \textbf{77.61} \\
M-Gate & 1.00 & 1.01 & \textbf{1.12} & 0.00 & 0.00 & 0.00 & \textbf{96.98} & \textbf{96.98} & 96.99 & 25.00 & 24.82 & \textbf{16.43} & 1.17 & \textbf{1.05} & 1.36 & 17.10 & \textbf{17.09} & \textbf{17.09} \\
Perm & 1.00 & 1.03 & \textbf{1.08} & 0.00 & 0.00 & 0.00 & \textbf{92.34} & 92.58 & 92.62 & 10.55 & \textbf{9.31} & 10.52 & 5.84 & 6.25 & \textbf{5.69} & 24.02 & 24.14 & \textbf{14.40} \\
P-Attn & 1.00 & 1.01 & \textbf{1.08} & 16.40 & 16.30 & \textbf{11.39} & 91.23 & \textbf{90.29} & 90.33 & \textbf{8.26} & 8.30 & 8.95 & 3.94 & \textbf{3.81} & 4.11 & 363.81 & 360.76 & \textbf{310.12} \\
Resize & 1.00 & 0.97 & \textbf{1.18} & 9.83 & 9.98 & \textbf{0.00} & 99.50 & 99.51 & \textbf{99.42} & 16.45 & 17.38 & \textbf{13.87} & \textbf{1.76} & 2.08 & 2.15 & 56.84 & 56.84 & \textbf{37.54} \\
\bottomrule
\end{tabular}%
}
\end{table}

\subsubsection{Optimizing Dynamic-Shape Operators for NPU Library} 

% % \begin{table*}[htb]
%     \centering
%     \caption{Performance metrics of attention \ding{175} for both Prefill and Decode on the NPU server.}
%     \label{tab:op-ratio-summary}
%     \renewcommand{\arraystretch}{1}
%     \begin{adjustbox}{max width=\textwidth}
%     \begin{tabular}{@{} c c *{3}{c} c @{}}
%         \toprule
%         \textbf{OP} & \textbf{Compilation Approach} & \textbf{Kernel Latency (us)} & \textbf{Scalar Computation Time (us)} & \textbf{HBM Data Read Time (us)} & \textbf{Compile Time (s)} \\
%         \midrule
%         \multirow{3}{*}{DFA} & Dynamic & 27.2 & 9.6 & 5.2 & 19.3 \\
%                              & JIT & 20.6 ($\downarrow 6.2$) & 7.1 ($\downarrow 2.5$) & 4.3 ($\downarrow 0.9$) & $19.4 \times |S|$  \\
%                              & \mysys & 22.6 ($\downarrow 4.2$) & 8.5 ($\downarrow 1.1$) & 4.3 ($\downarrow 0.9$) & 44.9 ($\uparrow 25.6$)\\
%         \midrule
%         \multirow{3}{*}{PFA} & Dynamic & 560.0 & 501.9 & 387.5 & 34.7 \\
%                              & JIT & 537.1 ($\downarrow 22.9$) & 485.6 ($\downarrow 16.3$) & 369.5 ($\downarrow 18.0$) & $34.7 \times |S|$ \\
%                              & \mysys & 541.8 ($\downarrow 18.2$) & 488.8 ($\downarrow 13.1$) & 369.0 ($\downarrow 18.5$) & 107.1 ($\uparrow 72.4$) \\
%         \bottomrule
%     \end{tabular}
%     \end{adjustbox}
% \end{table*}


\begin{table*}[htb]
    \centering
    \caption{Detailed performance metrics of attention \ding{176} for both Prefill and Decode on the NPU server. We profile using $B=4, S_{qo} = 1024$ and $S_{kv} = S_{qo}$ for Prefill. The $\uparrow$($\downarrow$) arrow in the parentheses denote the absolute increase (reduction) compared to the dynamic approach.}
    \label{tab:detailed-op-metric-NPU}
    \renewcommand{\arraystretch}{1}
    \begin{adjustbox}{max width=\textwidth}
    \begin{tabular}{@{} c c *{3}{c} c @{}}
        \toprule
        \textbf{OP} & \textbf{Compilation Approach} & \textbf{Latency (us)} & \textbf{Scalar Time (us)} & \textbf{HBM Access Time (us)} & \textbf{Cube Time (s)} \\
        \midrule
        \multirow{3}{*}{DFA} & Dynamic & 27.2 & 10.2 & 7.5 & 2.18 \\
                             & JIT & 20.6 ($\downarrow 6.6$) & 7.8 ($\downarrow 2.4$) & 5.7 ($\downarrow 1.8$) & 2.18 ($0$)  \\
                             & \mysys & 22.6 ($\downarrow 4.6$) & 9.2 ($\downarrow 1$) & 5.8 ($\downarrow 1.7$) & 2.18 ($0$) \\
        \midrule
        \multirow{3}{*}{PFA} & Dynamic & 560.0 & 501.9 & 387.5 & 125.7 \\
                             & JIT & 537.1 ($\downarrow 22.9$) & 485.6 ($\downarrow 16.3$) & 369.5 ($\downarrow 18.0$) &  125.7 ($0$) \\
                             & \mysys & 541.8 ($\downarrow 18.2$) & 488.8 ($\downarrow 13.1$) & 369.0 ($\downarrow 18.5$) &  125.7 ($0$)\\
        \bottomrule
    \end{tabular}
    \end{adjustbox}
\end{table*}


\begin{figure}[H]
    \centering
    \includegraphics[width=0.9\linewidth]{figures/evaluation/speedup-NPU/prefill_decode_comparison_chart.png}
    \caption{Kernel optimization for different types of attentions in \autoref{tab:benchmark-setting-attention-types} on NPU server. }
    \label{fig:NPU-kernel-opt-ave}
\end{figure}

\autoref{fig:NPU-kernel-opt-ave} shows the overall speedups of \mysys compared to the unspecialized kernels in the NPU library~\cite{ascend_cannopsadv} with both DFA and PFA operators.

Notably, the DFA operator achieved approximately 12\% acceleration through our \mysys optimization, primarily attributed to initialization optimization similar to that illustrated in \autoref{fig:pipeline}.
During the initialization phase, the DFA operator involves numerous configurable parameters that mainly require value assignments with minimal computational operations.

The instruction latency for computational operations typically remains at 1 cycle, while load/store operations associated with assignments exhibit significantly higher latency. 
Particularly in cases of cache misses, the latency for load/store operations can escalate to hundreds of cycles. 
Through TilingInfer optimization, certain parameters can be optimized into constants, thereby eliminating the corresponding load/store operations and achieving substantial performance improvement.
This optimization mechanism effectively reduces memory access overhead by converting dynamic parameter assignments into static constant values during compilation.

Regarding the PFA operators, it shows only an average of 3\% speedup ratio.
\autoref{tab:detailed-op-metric-NPU} illustrates why there are significantly different speedup ratios.
Notably, while the PFA operator demonstrates substantial absolute latency reduction (560$\mu s\to 541.8\mu s$) through initialization-phase load/store elimination - a mechanism analogous to the DFA optimization, its relative improvement remains constrained (3.3\% vs. DFA's 20.3\%). 
This phenomenon stems from PFA's computational and memory intensity during prefill phase execution, where initialization overhead is only a small fraction of the total runtime. 

% \begin{table*}[htb]
%     \centering
%     \caption{Performance metrics of attention \ding{175} for both Prefill and Decode on the NPU server.}
%     \label{tab:op-ratio-summary}
%     \renewcommand{\arraystretch}{1}
%     \begin{adjustbox}{max width=\textwidth}
%     \begin{tabular}{@{} c c *{3}{c} c @{}}
%         \toprule
%         \textbf{OP} & \textbf{Compilation Approach} & \textbf{Kernel Latency (us)} & \textbf{Scalar Computation Time (us)} & \textbf{HBM Data Read Time (us)} & \textbf{Compile Time (s)} \\
%         \midrule
%         \multirow{3}{*}{DFA} & Dynamic & 27.2 & 9.6 & 5.2 & 19.3 \\
%                              & JIT & 20.6 ($\downarrow 6.2$) & 7.1 ($\downarrow 2.5$) & 4.3 ($\downarrow 0.9$) & $19.4 \times |S|$  \\
%                              & \mysys & 22.6 ($\downarrow 4.2$) & 8.5 ($\downarrow 1.1$) & 4.3 ($\downarrow 0.9$) & 44.9 ($\uparrow 25.6$)\\
%         \midrule
%         \multirow{3}{*}{PFA} & Dynamic & 560.0 & 501.9 & 387.5 & 34.7 \\
%                              & JIT & 537.1 ($\downarrow 22.9$) & 485.6 ($\downarrow 16.3$) & 369.5 ($\downarrow 18.0$) & $34.7 \times |S|$ \\
%                              & \mysys & 541.8 ($\downarrow 18.2$) & 488.8 ($\downarrow 13.1$) & 369.0 ($\downarrow 18.5$) & 107.1 ($\uparrow 72.4$) \\
%         \bottomrule
%     \end{tabular}
%     \end{adjustbox}
% \end{table*}


\begin{table*}[htb]
    \centering
    \caption{Detailed performance metrics of attention \ding{176} for both Prefill and Decode on the NPU server. We profile using $B=4, S_{qo} = 1024$ and $S_{kv} = S_{qo}$ for Prefill. The $\uparrow$($\downarrow$) arrow in the parentheses denote the absolute increase (reduction) compared to the dynamic approach.}
    \label{tab:detailed-op-metric-NPU}
    \renewcommand{\arraystretch}{1}
    \begin{adjustbox}{max width=\textwidth}
    \begin{tabular}{@{} c c *{3}{c} c @{}}
        \toprule
        \textbf{OP} & \textbf{Compilation Approach} & \textbf{Latency (us)} & \textbf{Scalar Time (us)} & \textbf{HBM Access Time (us)} & \textbf{Cube Time (s)} \\
        \midrule
        \multirow{3}{*}{DFA} & Dynamic & 27.2 & 10.2 & 7.5 & 2.18 \\
                             & JIT & 20.6 ($\downarrow 6.6$) & 7.8 ($\downarrow 2.4$) & 5.7 ($\downarrow 1.8$) & 2.18 ($0$)  \\
                             & \mysys & 22.6 ($\downarrow 4.6$) & 9.2 ($\downarrow 1$) & 5.8 ($\downarrow 1.7$) & 2.18 ($0$) \\
        \midrule
        \multirow{3}{*}{PFA} & Dynamic & 560.0 & 501.9 & 387.5 & 125.7 \\
                             & JIT & 537.1 ($\downarrow 22.9$) & 485.6 ($\downarrow 16.3$) & 369.5 ($\downarrow 18.0$) &  125.7 ($0$) \\
                             & \mysys & 541.8 ($\downarrow 18.2$) & 488.8 ($\downarrow 13.1$) & 369.0 ($\downarrow 18.5$) &  125.7 ($0$)\\
        \bottomrule
    \end{tabular}
    \end{adjustbox}
\end{table*}


% \begin{figure*}[htp]
%     \centering
%     \includegraphics[width=\textwidth]{figures/evaluation/overall_benefits.pdf}
%     \caption{Runtime latency comparison of the operator DFA using different compile approaches. Although the compile method using static shape information achieves the least kernel latency, the Just-In-Time (JIT) compile overhead leads to minutes of latency, making it impractical for dynamic shape situations.}
%     \label{fig:NPU-benefits}
% \end{figure*}

\subsubsection{Optimizing Dynamic-Shape Operators for GPU Library}


\begin{table}[ht]
\centering
\setlength{\tabcolsep}{3pt}
\begin{threeparttable}

\caption{Code size reduction and kernel optimization results for different types of attentions in \autoref{tab:benchmark-setting-attention-types} on GPU server. The performance baseline is FlashInfer~\cite{ye2025flashinfer}, which specializing kernels in the manual dispatch style.
The original code size is measured using the value domain listed in \autoref{tab:attention-dimensions-definition}.
}
\label{tab:dimensions-value-domain}
\footnotesize
\begin{tabularx}{0.95\linewidth}{ c c c l X }
\toprule
\textbf{ID} & \textbf{Code Size(MB)} & \textbf{Reduction Ratio} & \textbf{Op} & \textbf{Speedup Ratio} \\
\midrule
\multirow{2}{*}{\ding{172}} & \multirow{2}{*}{19.2($\downarrow 501.3$)} & \multirow{2}{*}{96.3\%} & DFA & 1.02× \\
 & & & DFA & 1.00× \\
\hline
\multirow{2}{*}{\ding{173}} & \multirow{2}{*}{20.2($\downarrow 500.3$)} & \multirow{2}{*}{96.2\%} & DFA &1.03× \\
 & & & DFA & 1.01× \\
\hline
\multirow{2}{*}{\ding{174}} & \multirow{2}{*}{18.9($\downarrow 501.6$)} & \multirow{2}{*}{96.4\%} & DFA & 1.03× \\
 & & & DFA & 1.01× \\
\hline
\multirow{2}{*}{\ding{175}} & \multirow{2}{*}{19.1($\downarrow 501.4$)} & \multirow{2}{*}{96.3\%} & DFA & 1.02× \\
 & & & DFA & 1.00× \\
\hline
\multirow{2}{*}{\ding{176}} & \multirow{2}{*}{22.1($\downarrow 498.4$)} & \multirow{2}{*}{95.8\%} & DFA & 1.03× \\
 & & & DFA & 1.02× \\
\hline
\multirow{2}{*}{\textbf{Average}} & \multirow{2}{*}{19.9($\downarrow 500.60$)} & \multirow{2}{*}{96.20\%} & DFA & 1.03× \\
 & & & DFA & 1.01× \\
\bottomrule
\end{tabularx}

\end{threeparttable}
\end{table}
% 520.5

We applied \mysys to specialize operators in a production-level GPU library~\cite{ye2025flashinfer}, which is the default operator backend for many LLM serving systems such as vLLM~\cite{kwon2023vllm} and SGLang~\cite{zheng2024sglang}.

\mysys tries to address both the significant code size inflation from manual dispatching and the performance of the kernels. 
The aggregated results are summarized in \autoref{tab:dimensions-value-domain}.
Our method achieves a remarkable 96.2\% average reduction in code size, compressing the compiled shared library by approximately 500.6 MB.
This demonstrates the effectiveness of \mysys in mitigating library bloat.

In terms of performance, \mysys delivers consistent, positive speedups over the already optimized baseline kernels.
On average, we observe a 1.03$\times$ speedup for PFA and a 1.01$\times$ speedup for DFA. 
The moderate performance gain is expected, as the baseline manual dispatch approach already specializes the most critical performance parameters.

To provide a more detailed view, we use operator \ding{176} as a case study to analyze the performance characteristics across different input sequence lengths, as illustrated in \autoref{fig:GQA-prefill} and \autoref{fig:GQA-decode}. 
We observe that the source and magnitude of performance gains vary between operator types.
Speedup comparison of
For the PFA (\autoref{fig:GQA-prefill}), our specialized kernel shows performance advantages, especially for shorter sequence lengths. 
We measured an average speedup of 1.04$\times$ for sequence lengths under 1k. 
In contrast, the DFA (\autoref{fig:GQA-decode}) exhibits more modest gains. 
The speedup is marginal, averaging 1.01$\times$ for sequence lengths over 1k and showing negligible improvement for shorter ones, occasionally performing on par with the baseline.
We attribute this difference to the inherent characteristics of the kernels; the PFA has a larger set of kernel parameters that benefit from specialization, whereas the DFA is simpler with fewer such parameters, as detailed in \autoref{tab:tiling-params-statistics}.

\begin{figure}[htbp]
    \centering
    \begin{subfigure}[b]{\linewidth}
        \centering
        \includegraphics[width=0.9\linewidth]{figures//evaluation/comparison_bench_flashinfer_batch_prefill_half_half_speedup.pdf}
        \caption{PFA kernel performance speedup compared to the baseline.}
        \label{fig:GQA-prefill}
    \end{subfigure}
    \vspace{0.5em}
    \begin{subfigure}[b]{\linewidth}
        \centering
        \includegraphics[width=0.9\linewidth]{figures//evaluation/comparison_bench_flashinfer_batch_decode_half_half_speedup.pdf}
        \caption{DFA kernel performance speedup compared to the baseline.}
        \label{fig:GQA-decode}
    \end{subfigure}
    \caption{Speedup of attention \ding{176} across different input sequence lengths. We use $B=4$.}
    \label{fig:gqa-comparison}
\end{figure}

\subsubsection{Acceleration Ratio vs. Key Dimensions Settings}

\begin{figure}[ht]
    \centering
    \includegraphics[width=1\linewidth]{figures//evaluation/combined_jit_p_speedup.png}
    \caption{Specialization ratio of the MQA attention operators using four different shape parameters across varying sequence lengths on NPU server.}
    \label{fig:ifa-4shape}
\end{figure}


In this section, we analyze the impact of key dimensional parameters on the performance gains achieved by constant specialization.
We focus our analysis on NPU operators, as their baseline provides a clearer opportunity to demonstrate these effects compared to the already heavily optimized GPU baseline kernels, where similar gains were more modest.
We conducted experiments on an MQA-based DFA operator, specializing kernels using different shape parameters batch size $B$ and the number of queries $N_{q}$, while keeping the head dimension $D$ fixed at 512. 
The results are presented in \autoref{fig:ifa-4shape}.
A primary observation is that the speedup ratio consistently decreases as the query sequence length increases.
Constant specialization yields a relatively fixed amount of time savings by optimizing the kernel's initialization stage.
As the sequence length grows, the workload strength increases substantially. 
Consequently, the proportion of time saved by our fixed optimization becomes smaller relative to the total execution time.

Furthermore, the parameters that define the overall workload, $B$ and $N_{q}$, also influence the speedup ratio.
As shown, configurations with larger workloads (e.g., $B=32, N_{q}=128$) yield a lower relative speedup compared to those with smaller workloads (e.g., $B=16, N_{q}=32$). 
This is due to the same amortization effect: the constant-time optimization benefit is spread over a larger total workload, thus reducing its relative impact.

Despite this amortization effect, our method demonstrates considerable effectiveness on NPU operators. 
For sequence lengths under 2k, the average speedup remains significant at approximately 1.10$\times$, confirming the practical benefits of our approach for NPU kernel optimization.

\subsection{Effectiveness Analysis of \mysys{} (\textbf{RQ2})}

\begin{table}[htb]
    \centering
    \caption{The compile time overhead and compare effectiveness of \mysys on NPU servere.}
    \label{tab:op-ratio-summary}
    \renewcommand{\arraystretch}{1}
    \begin{adjustbox}{max width=\linewidth}
    \begin{tabular}{@{} c c c c @{}}
        \toprule
        \textbf{OP} & \textbf{Compilation Approach} & \textbf{Compile Time (s)} & \textbf{Speedup over Dynamic} \\
        \midrule
        \multirow{3}{*}{DFA} & Dynamic & 19.4 & 1.00 \\
                             & JIT     & $19.4 \times |S_{kv}|$ & 1.14 \\
                             & \mysys  & 20.6($\uparrow 1.2$) & 1.12  \\
        \midrule
        \multirow{3}{*}{PFA} & Dynamic & 21.9 & 1.00 \\
                             & JIT     & $21.9 \times |S_{kv}|$ & 1.03 \\
                             & \mysys  & 23.2 ($\uparrow 1.3$) & 1.03  \\
        \bottomrule
    \end{tabular}
    \end{adjustbox}
\end{table}

The \autoref{tab:op-ratio-summary} compares the compilation overhead of each approach. 
The Dynamic approach compiles only once, yielding the lowest compilation time but suboptimal performance. 
In contrast, the JIT approach generates a specialized kernel for each of the $|S_{kv}|$ shape instances. While this achieves optimal performance, its compilation overhead becomes impractically high when many instances exist.
\mysys{} introduces only a minimal overhead. Its total time comprises the original kernel compilation (19.4s for DFA and 21.9s for PFA) plus a small constant propagation cost (1.2s for DFA and 1.3s for PFA), increasing the average compile time by a modest 6.\%.

Despite this low compilation overhead, \mysys{} delivers performance highly competitive with the theoretical optimum of JIT compilation. 
As shown in the \autoref{tab:op-ratio-summary}, \mysys{} achieves a 12\% speedup through kernel specialization, reaching 85.7\% of the 14\% speedup provided by JIT.
This narrow performance gap indicates that our method effectively specializes the most performance-critical constants. 
The marginal potential for further gains does not justify adopting more computationally expensive techniques, such as constraint-guided symbolic execution, which would introduce significant compilation overhead.