\section{Implementation Details}
\label{sec:details}

% \mysys 基于 LLVM v15.0.5 搭建。\mysys 的实现整体上分为两部分:
% \item 针对算子的 tiling function ,\mysy 将每个算子的 tiling function 作为分析的入口函数。将算子的常量参数(例如常量维度、底层内存格式配置等)作为外部常量信息引入。
% 当前,算子的参数信息通常是通过 Python 侧 API 调用运行时传递进来,但动态语言的不确定性限制了我们直接从 Python 程序自动分析获得常量信息。我们的实现方法是,为每种算子的 tiling function 维护一个函数签名,外部常量信息通过配置文件传递给 LLVM 编译器,由编译器负责将其转换成 tiling function 的入参。

%  \item 针对核函数的特化,对于不同硬件平台实现方法不太相同。针对 Ascend NPU,我们基于 Bisheng 编译器实现了基于 LLVM Bitcode 的参数特化 Pass,之后递归调用诸如 IPSCCP, CSE 等优化后生成特化的核函数二进制文件。
% 对于 GPU 算子,由于 NVCC 编译器闭源且没有尝试基于 LLVM 编译核函数,我们的实现方式是识别出常量核函数参数后将其定义为模板参数,之后均通过模板参数获取该该参数。
% 此外,针对给定的模型形状参数取值,我们基于推理出来的常量参数,消除冗余分派,最终消除了完全未被调用的特化核函数,打包为更精简的共享库,从而实现了代码消减。

\mysys is built on LLVM 15.0.5 and optimizes machine learning operators and kernels by treating some shape parameters as compile-time constants. 
While these operators are typically invoked from high-level Python APIs, analyzing such dynamic code is notoriously complex. 
\mysys sidesteps this issue.
By using a configuration file to define an operator's constant parameters (e.g., invariable dimensions, memory layout), it specializes the underlying C++ implementation directly, starting from its core tiling function.

Our approach to specializing the low-level kernel functions varies by hardware:
\begin{itemize}
    \item \textbf{For Ascend NPUs:} We use the Bisheng compiler to perform specialization on the LLVM bitcode.
    \item \textbf{For NVIDIA GPUs:} Since the NVCC compiler is closed-source, we convert the constant parameters into C++ template parameters to generate specialized versions of the kernels.
\end{itemize}

Finally, the compiler eliminates any specialized functions that are not used for a given model. 
This process removes redundant code, resulting in a final, streamlined shared library with a significantly smaller binary size.
